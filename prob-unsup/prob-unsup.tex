\documentclass{article}
\usepackage[utf8]{inputenc}
\usepackage{amsmath}
\usepackage{amsthm}
\usepackage{amsfonts}
\usepackage[colorlinks]{hyperref}
\usepackage{natbib}
\usepackage{graphicx}
\usepackage{algorithm} 
\usepackage{algpseudocode} 
\usepackage{booktabs}
\usepackage{caption}
\usepackage{tikz}
\usepackage{chngpage}
\usepackage{xcolor}
\usepackage{cancel}

\newtheorem{theorem}{Theorem}[section]
\newtheorem{corollary}{Corollary}[section]
\newtheorem{proposition}{Proposition}[section]
\newtheorem{lemma}{Lemma}[section]
\newtheorem{claim}{Claim}[section]
\newtheorem{conjecture}{Conjecture}[section]
\newtheorem{example}{Example}[section]

\theoremstyle{definition}
\newtheorem{definition}{Definition}[section]
 
\theoremstyle{remark}
\newtheorem{remark}{Remark}


\newcommand{\Phu}[1]{{\bf \color{red} [[Phu: #1]]}}
\setlength\parindent{0pt}
\setlength\parskip{5pt}
\usepackage[margin=1.0in]{geometry}

\newcommand{\dby}{\ \mathrm{d}}
\newcommand{\argmax}[1]{\underset{#1}{\arg\max \ }}
\newcommand{\argmin}[1]{\underset{#1}{\arg\min \ }}
\newcommand{\const}{\text{const.}}
\newcommand{\bracka}[1]{\left( #1 \right)}
\newcommand{\brackb}[1]{\left[ #1 \right]}
\newcommand{\brackc}[1]{\left\{ #1 \right\}}
\newcommand{\brackd}[1]{\left\langle #1 \right\rangle}
\newcommand{\abs}[1]{\left| #1 \right|}
\newcommand{\contractop}{\mathcal{B}}
\newcommand*\circled[1]{\tikz[baseline=(char.base)]{
            \node[shape=circle,draw,inner sep=2pt] (char) {#1};}}
\newcommand{\red}[1]{{\color{red} #1}}
\newcommand{\loss}{\mathcal{L}}
\newcommand{\correctquote}[1]{``#1''}
\newcommand{\norm}[1]{\left\lVert#1\right\rVert}

% From https://tex.stackexchange.com/questions/194426/split-itemize-into-multiple-columns
\usepackage{etoolbox,refcount}
\usepackage{multicol}

\newcounter{countitems}
\newcounter{nextitemizecount}
\newcommand{\setupcountitems}{%
  \stepcounter{nextitemizecount}%
  \setcounter{countitems}{0}%
  \preto\item{\stepcounter{countitems}}%
}
\makeatletter
\newcommand{\computecountitems}{%
  \edef\@currentlabel{\number\c@countitems}%
  \label{countitems@\number\numexpr\value{nextitemizecount}-1\relax}%
}
\newcommand{\nextitemizecount}{%
  \getrefnumber{countitems@\number\c@nextitemizecount}%
}
\newcommand{\previtemizecount}{%
  \getrefnumber{countitems@\number\numexpr\value{nextitemizecount}-1\relax}%
}
\makeatother    
\newenvironment{AutoMultiColItemize}{%
\ifnumcomp{\nextitemizecount}{>}{3}{\begin{multicols}{2}}{}%
\setupcountitems\begin{itemize}}%
{\end{itemize}%
\unskip\computecountitems\ifnumcomp{\previtemizecount}{>}{3}{\end{multicols}}{}}


\title{Probabilisitic and Unsupervised Learning}
\author{Phu Sakulwongtana}
\date{}

\begin{document}

\maketitle

\section{Graphical Model}

\subsection{Introduction}

\begin{definition}{\textbf{(Types of Graph)}}
    There are several kind of graphs that we can use to model the probability distribution: factor graph, undirected graph, and directed graph. Node corresponds to the random variables and the edge in graph indicates statisical dependence between varible.
\end{definition}

\begin{definition}{\textbf{(Dependencies)}}
    For the random variable $X,Y,V$, where we have:
    \begin{itemize}
        \item \emph{Conditional Independence}: $X\ind Y | V$ iff $P(X | Y, V) = P(X|V)$ provided that $P(Y, V)>0$. We can see that furthermore that:
        \begin{equation*}
            P(X, Y | V) = P(X | Y, V) P(Y| V) = P(X|V)P(Y|V)
        \end{equation*}
        Please note that, this can generalize the symbol to the sets of random variables as:
        \begin{equation*}
            \mathcal{X}\ind\mathcal{Y} | \mathcal{V} = \brackc{X\ind Y | \mathcal{V} : \forall X \in \mathcal{X}, \forall Y \in \mathcal{Y}}
        \end{equation*}
        \item \emph{Marginal Independence}: $X\ind Y$ is equivalent to $X\ind Y | \emptyset$ and $P(X, Y) = P(X)P(Y)$
    \end{itemize} 
\end{definition}

\begin{definition}{\textbf{(Factor Graph)}}
    Factor Graph is a directed graphical representation of the factorized model structure, where each square indicates the factor over the linked variables:
    \begin{equation*}
        P(\mathcal{X}) = \frac{1}{Z}\prod_j f_j(\mathcal{X}_{C_j})
    \end{equation*} 
    where we have the following components:
    \begin{itemize}
        \item $\mathcal{X} = \brackc{X_1,\dots,X_k}$
        \item $\mathcal{X}_S = \brackc{X_i : i \in S}$
        \item $j$ is index that indicates the factor $C_j$ that contains all indicies of variable adjecent to factor $j$
        \item $f_j$ is factor function
        \item $Z$ is normalization constant
    \end{itemize}
    The conditional independent is defined by $X\ind Y | \mathcal{V}$ if every path between $X$ and $Y$ contains some $V\in\mathcal{V}$ (this can be shown that).
\end{definition}

\begin{remark}{\textbf{(Conditional Distribution)}}
    Now, if every path between $X$ and $Y$ contains some $V \in \mathcal{V}$, then there exists a factorization. We have the following joint distribution
    \begin{equation*}
        P(X, Y, \mathcal{V},\dots) = \frac{1}{Z}g_X(X,\mathcal{V}_X,\mathcal{Q}_X) g_Y(Y, \mathcal{V}_Y, \mathcal{Q}_Y)g_R(\mathcal{Q}_R,\mathcal{V}_R)
    \end{equation*}
    where $\mathcal{V}_X,\mathcal{V}_Y,\mathcal{V}_R\subseteq\mathcal{V}$ and the set containing $\mathcal{Q}_X, \mathcal{Q}_Y, \mathcal{Q}_R$ are disjoint. The conditonal is:
    \begin{equation*}
    \begin{aligned}
        P(X|Y, \mathcal{V},\dots) &= \frac{P(X, Y, \mathcal{V},\dots)}{P(Y, \mathcal{V},\dots)} = \frac{\frac{1}{Z}g_X(X,\mathcal{V}_X,\mathcal{Q}_X) g_Y(Y, \mathcal{V}_Y, \mathcal{Q}_Y)g_R(\mathcal{Q}_R,\mathcal{V}_R)}{\sum_{X'}\frac{1}{Z}g_X(X,\mathcal{V}_X,\mathcal{Q}_X) g_Y(Y, \mathcal{V}_Y, \mathcal{Q}_Y)g_R(\mathcal{Q}_R,\mathcal{V}_R)} \\
        &= \frac{g_X(X,\mathcal{V}_X,\mathcal{Q}_X)}{\sum_{X'}g_X(X',\mathcal{V}_X,\mathcal{Q}_X)}
    \end{aligned}
    \end{equation*}
    One the RHS doesn't depend on $Y$ as it follows that $X\ind Y | \mathcal{V}$. 
\end{remark}

\begin{definition}{\textbf{(Markov Blanket)}}
    $\mathcal{V}$ is markov blanket for $X$ iff $X\ind Y | \mathcal{V}$ for all $Y\not\in\brackc{X\cup\mathcal{V}}$
\end{definition}

\begin{remark}
    Each variable $X$ is conditionally independent of all non-neighbourhood given its neighbourhood as we have:
    \begin{equation*}
        X \ind Y | \operatorname{ne}(X) \qquad \forall Y \not\in \brackc{X \cup \operatorname{ne}(X)}
    \end{equation*}
    All neighbourhood $\operatorname{ne}(X)$ is markov blanket of $X$. Please note that it is minimal of such set (markov blanket), which is called \emph{markov boundary}. 
\end{remark}

\begin{definition}{\textbf{(Cliques)}}
    Cliques is fully connected subgraph, whiel the maximal clique is a clique that isn't contains in the other cliques. 
\end{definition}

\begin{definition}{\textbf{(Undirected Graphical Model)}}
    The undirected graphical model is a direct representation of conditional independent and nodes are connected iff they are conditonally dependent given all others. The joint probability factors over maximal clique $C_j$ of the graph is given by:
    \begin{equation*}
        P(\mathcal{X}) = \frac{1}{Z}\prod_jf_j(\mathcal{X}_{C_j})
    \end{equation*}
    We have the following dependencies properties:
    \begin{itemize}
        \item $X \ind Y | \mathcal{V}$ if every path between $X$ and $Y$ contains some node $V \in\mathcal{V}$
        \item Each variable $X$ is conditionally independent of all non-neighbour node given its neighbourhood nodes:
        \begin{equation*}
            X\ind Y | \text{ne}(X) \qquad Y \in \brackc{X \cup \operatorname{ne}(X)}
        \end{equation*}
        And so, the neighbours is a markov blanket. 
    \end{itemize}
\end{definition}

\begin{remark}{\textbf{(Factor Graph vs Undirected Model)}}
    Consider $3$ difference types of graph, we can see that each nodes has same neighbour: 
    \begin{figure}[H]
        \centering
        \includegraphics[width=8cm]{img/img1.png}
    \end{figure}  
    Each graph represents exactly the same conditional independent relationship. However, the maximal factorization differs, suppose we have for each variable $K$ possible values:
    \begin{itemize}
        \item $(a)$ can't distiguish between these (we will adopt the $(b)$ to be safe)
        \item $(b)$ has $2$ three-way factor. This is represented in $\mathcal{O}(K^3)$-size table. 
        \item $(c)$ has only pairwise factors. This is represented in $\mathcal{O}(K^2)$-size table. 
    \end{itemize}
    This means that the factor graphs have richer expressive power than undirected graphical models. But the factors can't be determined by testing for conditional independent.
\end{remark}

\begin{remark}{\textbf{(Limitation of Undirected Graphical Model)}}
    Undirected and Factor graph fails to capture the dependencies as the pair of variables that may be connected because they are some other variable that depends on them, for example:
    \begin{figure}[H]
        \centering
        \includegraphics[width=8cm]{img/img2.png}
    \end{figure}  
    If the ground is damped, it may suggest that it was rain, but if we see a sprinkler, then this explain away the damp, thus reduce the our belief of the rain into the prior. For example:
    \begin{equation*}
        R \ind S | \emptyset \quad \text{ but } \quad R\ind S | G
    \end{equation*}
    This is where there is difference between marginal and conditional independent. 
\end{remark}

\begin{definition}{\textbf{(DAG Graphical Model)}}
    A directed acyclic graphical model (DAG) represents a factorization of the joint probability distribution in terms of condtional:
    \begin{equation*}
        P(X_1,\dots,X_n) = \prod^n_{i=1}P(X_i | X_{\text{pa}(i)})
    \end{equation*} 
    where $\text{pa}(\cdot)$ is the parent of node $i$. DAG models are called Baysian network.
\end{definition}

\begin{proposition}
    The conditional Independence between the graph is more complicated than the undirected graph i.e $X\ind Y | \mathcal{V}$. If we consider every undirected path between $X$ and $Y$, the path is blocked by $\mathcal{V}$ if there is a node $V$ on the path such that:
    \begin{itemize}
        \item $V$ has convergnece arrows $\rightarrow V \leftarrow$ on the path and neighbour $V$ nor its descendent are in $\mathcal{V}$
        \item $V$ doesn't have convergnece arrow $\leftarrow V\rightarrow$ or $\rightarrow V \rightarrow$ and $V \in \mathcal{V}$ 
    \end{itemize}
    If all paths are blocked, then $\mathcal{V}$ is $D$-separated between $X$ and $Y$, and so $X\ind Y | \mathcal{V}$. Furthermore, the markov boundary to be:
    \begin{equation*}
        \brackc{\operatorname{pa}(X)\cup\operatorname{ch}(X)\cup\operatorname{pa}(\operatorname{ch}(X))}
    \end{equation*}
\end{proposition}
\begin{proof}
    We can see that the conditional independence of the directed graphical model (for example $A\ind B | \mathcal{D}$) can be modeled as the passing of \correctquote{ball}, in which $2$ variables ($A$, $B$) aren't independence if there is a way that a ball can be passed between them. We will mark the nodes in $\mathcal{V}$ as shaded. There are $10$ simple rules:
    \begin{figure}[H]
        \centering
        \includegraphics[width=10cm]{img/img3.png}
    \end{figure}  
    Most of the rules are straightforward to see why it is enforced. 
    \begin{itemize}
        \item The first column: Both variables are separated by a middle node, then both of the are independent of each other, thus unable to pass the \correctquote{ball} to each other, meaning that they are independence given the middle node. (This represents the divergence arrow rule)
        \item Similar explaination can be done in the second column of the image. (This also represents the divergence arrow rule)
        \item For the third column (explaining away), we can see that both of the nodes are independence given nothing, however, they becomes dependence once the middle node is shown. This is a reflection of the convergnece arrow rule. 
        \item Finally, the last 2 columns are boundary rule, which is also straightforward to see why it is enforced.
    \end{itemize}
    Thus, we have the reason why the rules above are used. 
\end{proof}

\begin{remark}{\textbf{(Differences Between DAG and Factor Graph)}}
    There are some types of graphs that DAG can represent its probability distribution, which is:
    \begin{figure}[H]
        \centering
        \includegraphics[width=3cm]{img/img4.png}
    \end{figure}  
    This is the only graph that that DAG can't represent. This is because there will always be 2 non-adjecent parent sharing the same child, which implies that:
    \begin{itemize}
        \item The variables are dependence in DAG 
        \item But independence in undirected graph.
    \end{itemize}
    On the other hand, no undirected or factor graph can represent the following DAG and only these:
    \begin{figure}[H]
        \centering
        \includegraphics[width=3cm]{img/img5.png}
    \end{figure}  
    This follows from the previous analysis on the explaining away and marginal independence.
\end{remark}

\begin{definition}{\textbf{(Family of Distribution)}}
    Each graph $\mathcal{G}$ implies a set of conditional independence statement: $\mathcal{C}(G) = \brackc{X_i\ind Y_i | \mathcal{Y}}$. Each set $\mathcal{C}$ defines a family of distribution that satisfies all statement in $\mathcal{C}$:
    \begin{equation*}
        P_{\mathcal{C}(G)} = \brackc{P(\mathcal{X}) : P(X_i, Y_i | \mathcal{V}) = P(X_i | \mathcal{V}_i)P(Y_i|\mathcal{Y}_i) \text{ for all } X_i\ind Y_i | \mathcal{V}_i \text{ in } \mathcal{C}  }
    \end{equation*}
    Similarly, we have family distribution in the functional form i.e:
    \begin{equation*}
        P_{G} = \brackc{P(\mathcal{X}) : \frac{1}{Z}\prod_j f_j(\mathcal{X}_{C_j}) \text{ for some non-negative function } f_j }
    \end{equation*}
\end{definition}

\begin{remark}{\textbf{(Family of Distributions)}}
    We can consider the following facts:
    \begin{itemize}
        \item For directed graph: $P_G = P_{\mathcal{C}(G)}$
        \item For undirected graph: $P_G = P_{\mathcal{C}(G)}$ if all distribution are positive i.e $P(\mathcal{X})>0$ for all variable of $\mathcal{X}$
        \item Factor graphs are more expressive as for every undirected graph $G_1$, there is a factor graph $G_2$ such that: $P_{G_1} = P_{G_2}$ but not in other direction.
    \end{itemize}
    Adding edge implies removing conditional independency statement and thus enlarging family of distribution.
\end{remark}

\begin{remark}
    For the next few propositions, we will consider difference kinds of graphical models, which can be shown to interchange with each others:
    \begin{figure}[H]
        \centering
        \includegraphics[width=8cm]{img/img6.png}
    \end{figure}  
\end{remark}

\begin{proposition}{\textbf{(Polytree $\boldsymbol \rightarrow$ Tree-Structured Factor Graph)}}
    For DAG that has more than one root, we consider:
    \begin{equation*}
        P(\mathcal{X}) = \prod_i P(X_i | X_{\text{pa}(i)}) =\prod_if_i(X_{C_i})
    \end{equation*} 
    where $C_i = i \cup \operatorname{pa}(i)$ and $f(X_{C_i}) = P(X_i | X_{\operatorname{pa}(i)})$
\end{proposition}

\begin{proposition}{\textbf{(Undirected Tree $\boldsymbol \rightarrow$ Factor Graph)}}
    Since all undirected tree have maximal clique of size $2$, it is equivalent to factor graph with pairwise factor:
    \begin{equation*}
        P(\mathcal{X}) = \frac{1}{Z}\prod_{\operatorname{edge}(i,j)}f_{(i,j)}(X_i, X_j)
    \end{equation*}
\end{proposition}

\begin{proposition}{\textbf{(Rooted Directed Tree $\boldsymbol \rightarrow$ Undirected Tree)}}
    The distribution for single rooted directed tree can be written as a product of pairwise factor of the undirected tree:
    \begin{equation*}
        P(\mathcal{X}) = P(X_r)\prod_{i\ne r}P(X_i | X_{\operatorname{pa}(i)}) = \prod_{\operatorname{edge}(i,j)}f_{(i,j)}(X_i, X_j)
    \end{equation*} 
\end{proposition}

\begin{proposition}{\textbf{(Undirected Tree $\boldsymbol \rightarrow$ Rooted Directed Tree)}}
    Choose arbitary node $X_r$ and set it as a root, which we will point every arrow away from it. Compute the conditional in the DAG as:
    \begin{equation*}
        P(\mathcal{X}) = P(X_r) \prod_{i\ne r}P(X_i|X_{\operatorname{pa}(i)}) = P(X_r)\prod_{i\ne r}\frac{P(X_i, X_{\operatorname{pa}(i)})}{P(X_{\operatorname{pa}(i)})} = \frac{\prod_{\operatorname{edge}(ij)}P(X_i,X_j)}{\prod_{\operatorname{nodes}(i)} P(X_i)^{\operatorname{deg}(i)- 1}}
    \end{equation*}
\end{proposition}

\section{Belief Propagation}

\begin{remark}
    We want to calculate the marginal distribution of a single node $P(X_i)$ and on the edge $P(X_i, X_j)$ based on the undirected graphical model, which we would need to use a belief propagation.
\end{remark}

\begin{proposition}{\textbf{(Marginal Distribution)}}
    The marginal distribution can be calculated locally as:
    \begin{equation*}
        P(X_i) = \prod_{X_j \in \operatorname{ne}(i)} M_{j\rightarrow i} \quad \text{ where } \quad M_{j\rightarrow i} = \sum_{X_j}f_{ij}(X_i, X_j) \prod_{X_k \in \operatorname{ne}(X_j)\backslash\brackc{X_i}} M_{k\rightarrow j}(X_j)
    \end{equation*}
\end{proposition}    
\begin{proof}
    For each neighbourhood $X_j$ of $X_i$ define a disjoint subtree $T_{j\rightarrow i}$ that split by $X_i$:
    \begin{equation*}
    \begin{aligned}
        P(X_i) &= \sum_{\mathcal{X}\backslash\brackc{X_i}} P(\mathcal{X}) \propto \sum_{\mathcal{X}\backslash\brackc{X_i}}\prod_{(i, j)\in \mathcal{E}_T} f_{ij}(X_{i}, X_{j}) \\
        &= \sum_{\mathcal{X}\backslash\brackc{X_i}}\prod_{X_j \in \operatorname{ne}(X_i)}f_{ij}(X_i, X_j)\prod_{(i', j')\in\mathcal{E}_{T_{j\rightarrow i}}}f_{i'j'}(X_{i'}, X_{j'}) \\
        &= \prod_{X_j \in \operatorname{ne}(i)} \bracka{\sum_{\mathcal{X}_{T_{j\rightarrow i}}} f_{ij}(X_i, X_j) \prod_{(i',j')\in\mathcal{E}_{T_{j\rightarrow i}}} f_{i'j'}(X_{i'}, X_{j'}) }
    \end{aligned}
    \end{equation*}
    The last equality comes from the splitting between subtrees, as they are disjoint. $\mathcal{X}_{T_{j\rightarrow i}}$ denotes each edge in the subtree. Let's consider the message:
    \begin{equation*}
    \begin{aligned} 
        M_{j\rightarrow i} 
        &= \sum_{\mathcal{X}_{T_{j\rightarrow i}}} f_{ij}(X_i, X_j) \prod_{(i',j')\in\mathcal{E}_{T_{j\rightarrow i}}} f_{i'j'}(X_{i'}, X_{j'})  \\ 
        &= \sum_{X_j} f_{ij}(X_i, X_j)\sum_{\mathcal{X}_{T_{j\rightarrow i}}\backslash \brackc{X_j}} \prod_{(i',j')\in\mathcal{E}_{T_{j\rightarrow i}}} f_{i'j'}(X_{i'}, X_{j'}) \\
        &= \sum_{X_j} f_{ij}(X_i, X_j) \prod_{X_{k}\in\operatorname{ne}(X_j)\backslash X_i} M_{k\rightarrow i}(X_j)
    \end{aligned}
    \end{equation*}
    The second equality comes from the fact that the factor of $X_i$ is the root of the tree and for any nodes that connection to the factors. If we consider:
    \begin{equation*}
    \begin{aligned}
        \sum_{\mathcal{X}_{T_{j\rightarrow i}}\backslash \brackc{X_j}} \prod_{(i',j')\in\mathcal{E}_{T_{j\rightarrow i}}} f_{i'j'}(X_{i'}, X_{j'}) &\propto P_{T_{j\rightarrow i}}(X_j) \\
        &\propto \prod_{X_{k}\in\operatorname{ne}(X_j)\backslash X_i} M_{k\rightarrow i}(X_j)
    \end{aligned}
    \end{equation*}
    This is due to recursive property of the message passing.
\end{proof}

\begin{proposition}{\textbf{(Pairwise Marginal)}}
    \begin{equation*}
        P(X_i, X_j) = f_{ij}(X_i,X_j)\prod_{X_k \in \operatorname{ne}
        (X_j)\backslash \brackc{X_i} }M_{k\rightarrow j}(X_j) \prod_{X_k\in\operatorname{ne}(X_i)\backslash\brackc{X_j}} M_{k\rightarrow i}(X_i)
    \end{equation*}
\end{proposition}
\begin{proof}
    We consider:
    \begin{equation*}
    \begin{aligned}
        P(X_i, X_j) &= \sum_{X\backslash\brackc{X_i, X_j}}P(\mathcal{X}) \propto \sum_{X\backslash\brackc{X_i, X_j}}\prod_{(i,j)\in\mathcal{E}_T}f_{ij}(X_i, X_j) \\
        &= \sum_{X\backslash\brackc{X_i, X_j}}f_{ij}(X_i, X_j)\prod_{(i'j')\in\mathcal{E}_{T_{j\rightarrow i}}}f_{i'j'}(X_{i'}, X_{j'})\prod_{(i'j')\in\mathcal{E}_{T_{i\rightarrow j}}}f_{i'j'}(X_{i'}, X_{j'}) \\
        &= f_{ij}(X_i, X_j)\bracka{\sum_{\mathcal{X}_{T_{j\rightarrow i}}} \prod_{(i'j')\in\mathcal{E}_{j\rightarrow i}} f_{i'j'}(X_{i'}, Y_{j'}) }\bracka{ \sum_{\mathcal{X}_{T_{i\rightarrow j}}}\prod_{(i'j')\in\mathcal{E}_{T_{i\rightarrow j}}} f_{i'j'}(X_{i'}, Y_{j'}) } \\
        &= f_{ij}(X_i,X_j)\prod_{X_k \in \operatorname{ne}
        (X_j)\backslash \brackc{X_i} }M_{k\rightarrow j}(X_j) \prod_{X_k\in\operatorname{ne}(X_i)\backslash\brackc{X_j}} M_{k\rightarrow i}(X_i)
    \end{aligned}
    \end{equation*}
\end{proof}

\begin{remark}{\textbf{(Belief Propagation for Inference)}}
    Let's consider the belief propagation for inference on the following graphical model:
    \begin{figure}[H]
        \centering
        \includegraphics[width=8cm]{img/img7.png}
    \end{figure}  
    To compute the $P(X_i | X_a = a)$ as we have the following message:
    \begin{equation*}
        M_{a\rightarrow i} = f_{ai}(X_a = a, X_i)
    \end{equation*}
    Please note that computing $P(X_i)$ requires that $M_{a\rightarrow i} = \sum_{X_a}f_{ai}(X_a, X_i)$. 
    For the internal node that partition the graph, like variable like $X_b$, we have the following message:
    \begin{equation*}
        M_{b\rightarrow j} = f_{bj}(X_b = b, X_j) \qquad M_{b\rightarrow k} = f_{bk}(X_b = b, X_k)
    \end{equation*}
    Please note that $M_{i\rightarrow j}$ are proportional to likelihood based on any observed variable (within message subtree) and possibly scaled by the prior (depends on factorization):
    \begin{equation*}
        M_{i\rightarrow j}(X_j) \propto P(\mathcal{X}_{T_{i\rightarrow j}}\cap\mathcal{O} | X_j)P(X_j)
    \end{equation*}
    If we consider the message to the observed node, then we have the likelihood. Keeepin all messages unnormalize and any marginal, then the normalizer is the likelihood. 
\end{remark}

\begin{remark}{\textbf{(BP Latent Chain Model)}}
    We consider the belief propagation in the latent chain model, which is a rooted directed tree, as we have the following graphical model:
    \begin{figure}[H]
        \centering
        \includegraphics[width=8cm]{img/img8.png}
    \end{figure}  
    We use the backward-forward algorithm is just belief propagation on this graph:
    \begin{itemize}
        \item $\alpha_t(i) = M_{s_{t-1}\rightarrow s_t}(s_t = i) \propto P(x_{1:t}, s_t)$ 
        \item $\beta_t(i) = M_{s_{t+1}\rightarrow s_t}(s_t = i)\propto P(x_{t+1: T} | s_t)$
    \end{itemize}
    where we can easily see that:
    \begin{equation*}
        \alpha_t(i)\beta_t(i) = \prod_{j\in\operatorname{ne}(s_t)} M_{j\rightarrow s_t}(s_t = i) \propto P(s_t = i | \mathcal{O}) 
    \end{equation*}
    The algorithm like BP extend the power of graphical model beyound just encoding of independent and factorization. A single derivation surves multiple models.
\end{remark}

\subsection{Junction Tree}

\begin{remark}{\textbf{(Inference on Graph)}}
    The graphical model sometimes, which is represented by an undirected graph isn't a tree as we would like to find the marginal probability of the single value. There are several strategies that we can use:
    \begin{itemize}
        \item Propagate the local message anyway and hope for the best. This is called \correctquote{loopy belief propagation}, which is an approximation technique. 
        \item Grouping the variable together with multi-variable nodes until the resulting graph, which is a tree as we are going consider this. 
    \end{itemize}
\end{remark}

\begin{remark}{\textbf{(Transforming the Graph)}}
    Consider to transform the graph into one that is easier to handle. As the original graph $G$ encodes a distribution $P(\mathcal{X})$ with a certain factorization or independent structure:
    \begin{itemize}
        \item Transformation from $G$ into an easy to handle $G'$ as it will be valid if $P(\mathcal{X})$ can be represented by $G'$
        \item Ensuring this, we need every step of the graph transformation only to remove condtional independence (never adding them).
        \item Making the family of possible encoding distribution groups grows or stay the same at each step. 
        \item The factor potential on the new graph $G'$ are built from those given on $G$ as to make sure it encodes the same distribution.  
    \end{itemize}    
\end{remark}

\begin{definition}{\textbf{(Junction Tree)}}
    A junction tree is a tree whose node and edges are labelled with set of variables. 
    \begin{itemize}
        \item Each node is represented by a cliques, edges are labelled by intersection of cliques called seperator. 
        \item The cliques contains all adjacent separator. 
        \item Furthermore, if $2$ cliques contain variable $X$, all cliques and separator on the path between $2$ cliques must contain $X$
    \end{itemize}
\end{definition}

\begin{definition}{\textbf{(Constructing Junction Tree)}}
    We consider the following step that transform DAG into a junction tree, which is shown in the figure below:
    \begin{figure}[H]
        \centering
        \includegraphics[width=8cm]{img/img9.png}
    \end{figure}  
    There are many process that we have to perform. 

    \textbf{DAG to Factor Graph:} The factor can be see as the conditional distribution of DAG via:
    \begin{equation*}
        P(\mathcal{X}) = \prod_iP(X_i|X_{\operatorname{pa}(i)}) = \prod_if_i(X_C)
    \end{equation*}
    where $C_i = i\cup \operatorname{pa}(i)$ and $f_i(X_{C_i}) = P(X_i|X_{\operatorname{pa}(i)})$. Marginal distribution on roots $P(X_r)$ absorbed into an adjacent factor.

    \textbf{Observation in Factor Graph:} Usually, inference target a posterior marginal given a set of observed values $P(X_l|\mathcal{O})$; for example, $P(A | D =\text{wet}, C = \text{rain})$. Modify the factor linked to the observed node, or add single factor adjacent to the observed node:
    \begin{equation*}
        f_D(D) = \begin{cases}
            1 &\text{ if } D = \text{wet} \\
            0 &\text{ otherwise }
        \end{cases} \qquad f_C(C) = \begin{cases}
            1 &\text{ if } C = \text{rain} \\
            0 &\text{ otherwise }
        \end{cases}
    \end{equation*}
    
    \textbf{Factor Graph to Undirected Graph:} The trasnformation from DAG to undirected graph is called moralization, where: Marry all parents of each node by adding an edge to connect them. We also drop arrows on all edges. 

    \textbf{Triangulate the Undirected Graph:} Want every factor of DAG must be contained within a maximal cliques of the undirected graph. We will have to perform the following modification:
    \begin{itemize}
        \item Replace each factor by an undirected cliques. 
        \item Construct the potential on each maximal clique by multiplying together factor potential that fall on it, and ensure each factor potential only appear once. 
    \end{itemize}
    To do this, we shall modify the graph as:
    \begin{itemize}
        \item We join the loop into cliques, which can be very inefficient. 
        \item Triangulation is performed to add edges to graph, so that every loop of size $\ge4$ has at least one chord. We will adding it recursively to ensure that the loop $\ge4$ has chords too. 
        \item The graph where every loop of size $\ge4$ has at least one chord is called chordal or triangulated. 
        \item Adding the edge removes conditional independencies, which enlarge the family of distribution. 
    \end{itemize}
    To find the triangulation is NP-complete problem, so we resort to heuristic called variable elimination. Let's consider the order of elimination as we have:
    \begin{equation*}
    \begin{aligned}
        P(X_{(n)}) &= \sum_{X_{\sigma(n-1)}}\cdots\sum_{X_{\sigma(1)}} P(\mathcal{X}) \\
        &= \frac{1}{Z}\sum_{X_{\sigma(n-1)}}\cdots\sum_{X_{\sigma(1)}} \prod_i f_i(\mathcal{X}_{C_i}) \\
        &= \frac{1}{Z}\sum_{X_{\sigma(n-1)}}\cdots\sum_{X_{\sigma(2)}}\prod_{j : C_j \ni \sigma(2)}f_j(\mathcal{X}_{C_j})\sum_{X_{\sigma(1)}} \prod_{i:C_i\ni\sigma(1)}f_i(\mathcal{X}_{C_i}) \\
        &= \frac{1}{Z}\sum_{X_{\sigma(n-1)}}\cdots\sum_{X_{\sigma(2)}}\prod_{j : C_j \ni \sigma(2)}f_j(\mathcal{X}_{C_j})f_\text{new}(\mathcal{X}_\text{new})
    \end{aligned}
    \end{equation*}
    Please note that $C_\text{new}$ is the neighbour of $X_{\sigma(1)}$ and \emph{edges are added to graph connected all nodes in} $C_\text{new}$:
    \begin{itemize}
        \item The graph including of all edges would be induced by elimination is chordal. 
        \item Finding a good triangulation depends on findinga good order of elimination $\sigma(1),\dots,\sigma(n)$. 
        \item It is NP-complete to find the best heuristic, as there are $2$ ways that we pick the next varaible to elimiate as follows:
        \begin{itemize}
            \item Min-Deficiency Search: Choose variable that induces the fewest new edge. 
            \item Max-Cardinal Search: Choose node with most previous visited neighbour.
        \end{itemize}
        In most experiments, min-deficiency search seem empirically be better. 
    \end{itemize}
    \textbf{Chordal Graph to Junction Tree:} To build a junction tree, we follows the procedure as we have:
    \begin{itemize}
        \item Find the maximal clique $C_1,\dots,C_k$ of the chordal undirected graph. 
        \item Create a weighted graph, which nodes are labelled by maximal cliques and edges connected each pair of cliques that shares variables. 
        \item Create an edge with size of separator as we find maximal weight spanning tree of weighted graph. 
    \end{itemize}
    Thus, we have the junction tree.
\end{definition}

\begin{remark}{\textbf{(Junction Tree Figures)}}
    The joint distribution factors over junction tree is:
    \begin{equation*}
        P(\mathcal{X})  = \frac{1}{Z}\prod_if_i(X_{C_i}) = \dots f_\text{ABC}(A, B, C)f_\text{BCD}(B, C, D)\dots
    \end{equation*}
    This violates the usual undirected tree sematics of factor per edge, and so we add the following constriants:
    \begin{itemize}
        \item Introducing the copy of the same variable so that there is no overlaps. 
        \item Adding new delta function that enforce consistency:
        \begin{equation*}
            P(\mathcal{X}) = \dots f_\text{ABC}(A, B^{(1)}, C^{(1)})\underbrace{\delta(B^{(1)} - B^{(2)})\delta(C^{(1)}, C^{(2)})}_{f_\text{sep}(B^{(1)}, C^{(1)}, B^{(2)}, C^{(2)})}f_\text{BCD}(B^{(2)}, C^{(2)}, D)\dots
        \end{equation*}
    \end{itemize}
    Having a new message passing the junction to be:
    \begin{itemize}
        \item Unshared Variable $X^{(-)}_{C_i} = X_{C_i \backslash \bigcup S_{ik}}$
        \item Variable incoming separator $X^{(2)}_{S_{ki}}$ (Same as matching variable $X^{(1)}_{S_{ki}}$ in $k\in\operatorname{ne}(i)\backslash j$)
        \item Variable outgoing separator $X^{(1)}_{S_{ij}}$ (Same as matching varaible $X^{(2)}_{S_{ij}}$ in clique $j$)
        \item The variable that appear in more than one separator will need additional copies.
    \end{itemize}
    The overall process is shown in the following junction tree figure:
    \begin{figure}[H]
        \centering
        \includegraphics[width=10cm]{img/img10.png}
    \end{figure}  
\end{remark}

\begin{definition}{\textbf{(Shafer-Shenopy)}}
    We have the following formula of the message:
    \begin{equation*}
    \begin{aligned}
        M_{i\rightarrow j}(X^{(2)}_{S_{ij}}) &= \sum_{ X^{(-)}_{C_i}, \brackc{X^{(2)}_{S_{ki}} }, S^{(+)}_{S_{ij}} } f_i\bracka{X^{(-)}_{C_i}, \brackc{X^{(2)}_{S_{ki}} }, S^{(+)}_{S_{ij}} } f_{ij}\bracka{X^{(1)}_{S_{ij}}, X^{(2)}_{S_{ij}}} \prod_{k\in\operatorname{ne}(i)\backslash j} M_{k\rightarrow i}(X^{(2)}_{S_{ij}}) \\
        &= \sum_{X_{C_i}\backslash S_{ij}}f_i(X_{C_i}) \prod_{k\in\operatorname{ne}(i)\backslash j} M_{k\rightarrow i}(X_{S_{ij}}) \\
    \end{aligned}
    \end{equation*}
    The marginal distribution on cliques and separator are defined by:
    \begin{equation*}
    \begin{aligned}
        P(X_{C_i}) \propto f_i(X_{C_i})\prod_{k\in\operatorname{ne}(C_i)}M_{k\rightarrow i}(X_{S_{ki}}) \qquad P(X_{S_{ij}}) \propto M_{i\rightarrow j}(X_{S_{ij}})M_{j\rightarrow i}(X_{S_{ij}})
    \end{aligned}
    \end{equation*}
\end{definition}

\begin{remark}{\textbf{(Junction Tree Properties)}}
    The running intersection property and tree structure of the junction tree implies that local consistency between cliques and separator marginal gurantee global consistency: If we consider the distribution $q_i(X_{C_i})$ and $r_{ij}(X_{S_{ij}})$ are distribution such that:
    \begin{equation*}
        \sum_{X_{C_i\backslash S_{ij}}} q_i(X_{C_i}) = r_{ij}(X_{S_{ij}})
    \end{equation*}
    This implies that that joint distribution to be:
    \begin{equation*}
        P(\mathcal{X}) = \frac{\prod_{\text{cliques } i} q_i(X_{C_i})}{\prod_{\text{separator } (ij)} r_{ij}(X_{S_{ij}})}
    \end{equation*}
    As we have:
    \begin{equation*}
        q_i(X_{C_i}) = \sum_{\mathcal{X}\backslash X_{C_i}}P(\mathcal{X}) \qquad r_{ij}(X_{S_{ij}}) =  \sum_{\mathcal{X}\backslash X_{S_{ij}}}P(\mathcal{X})
    \end{equation*}
\end{remark}

\begin{definition}{\textbf{(Hugin Update)}}
    Let's start by initializing the variables to be:
    \begin{equation*}
        q_i(X_{C_i}) \propto f_i(X_{C_i}) \qquad r_{ij}(X_{S_{ij}}) \propto 1
    \end{equation*}
    A Hugin propagation update for $i\rightarrow j$ is given by:
    \begin{equation*}
        r^\text{new}_{ij} = \sum_{X_{C_i\backslash S_{ij}}}q_i(X_{C_i}) \qquad q^\text{new}_j(X_{C_j}) = q_j(X_{C_j})\frac{r^\text{new}_{ij}(X_{S_{ij}})}{r_{ij}(X_{S_{ij}})}
    \end{equation*}
    Setting the correct marginalization locally for the first update. For the second update, we change the $q$ based on the keeping the joint probability $P(\mathcal{X})$
\end{definition}


\section{Estimation of Parameters}

\subsection{Method of Moments}

\begin{remark}
    Given the set of data $X_1,\dots,X_n$ sampled from a known distribution family but unknown parameter $P(x|\theta)$, we would like to this parameter.
\end{remark}

\begin{definition}{\textbf{(Moments)}}
    The $k$-th moment of probability is defined as $\mu_k = \mathbb{E}[X^k]$, where $X$ is random variable following distribution. The sample moment is:
    \begin{equation*}
        \hat{\mu}_k = \frac{1}{n}\sum^n_{i=1}X^k_i
    \end{equation*}
\end{definition}

\begin{definition}{\textbf{(Method of Moments)}}
    The method of moments estimates parameters by finding expression for them in terms of lowest possible order moments and substituting sample moments into the expression. Suppose there are $2$ parameters, which can be expressed in terms of $2$ moments as:
    \begin{equation*}
        \theta_1 = f_1(\mu_1,\mu_2)\qquad \theta_2=f_2(\mu_1,\mu_2)
    \end{equation*}
    Then method moments simply substitute the sample moment of the functions getting the parameter $\hat{\theta}_1, \hat{\theta}_2$.
\end{definition}

\begin{definition}{\textbf{(Sampling Distribution/Standard Error)}}
    It is natural question to aks to the distribution of the estimate, which is called \emph{sampling distribution}, or the approximation to that distribution. The standard error is the standard deviation of sampling distribution. 
\end{definition}

\begin{example}
    We will consider the use of method of moments in $3$ difference kinds of distribution:
    \begin{itemize}
        \item Poisson Distribution: This is simple as $\lambda = \mathbb{E}[X]$, so the parameter is set to:
        \begin{equation*}
            \hat{\lambda} = \bar{X} = \frac{1}{n}\sum^n_{i=1} X_i
        \end{equation*}
        To consider the sampling distribution, we have:
        \begin{equation*}
            p(\hat{\lambda} = n) = p(S = nv) = \frac{(n\lambda_0)^{nv}\exp(-n\lambda_0)}{(nv)!}
        \end{equation*}
        Since $S = \sum_iX_i$ is Poisson, the mean and variance are both $n\lambda_0$, so we have $\mathbb{E}[\hat{\lambda}] = 1/n \mathbb{E}[S] = \lambda_0$ and $\operatorname{var}(\hat{\lambda}) = \lambda_0/n$, and so the standard error is the square root of the variance.
        \item \textbf{Normal Distribution:} We can see that $\mathbb{E}[X] = \mu$ and $\mathbb{E}[X^2] = \mu^2 + \sigma^2$, and so, we have:
        \begin{equation*}
        \begin{aligned}
            &\hat{\mu} = \hat{\mu}_1 = \bar{X} \\
            &\hat{\sigma}^2 = \hat{\mu}_2 - \hat{\mu}_1^2 = \frac{1}{n}\sum^n_{i=1}X_i^2 - \bar{X}^2 
        \end{aligned}
        \end{equation*}
        We can see that the sampling distribution of $\bar{X}\sim\mathcal{N}(\mu, \sigma^2/n)$ and $n\hat{\sigma}^2/\sigma^2\sim\chi^2_{n-1}$. 
        \item \textbf{Gamma Distribution:} We can see that the first $2$ moments are given as $\mathbb{E}[X] = \alpha/\lambda$ and $\mathbb{E}[X^2] = (\alpha(\alpha + 1))/\lambda^2$. From the second equation: $\mu_2 = \mu_1^2 + \mu_1/\lambda$, and so we have:
        \begin{equation*}
            \hat{\lambda} = \frac{\hat{\mu_1}}{\hat{\mu}_2 - \hat{\mu}_1^2} \qquad\qquad \hat{\alpha} = \frac{\hat{\mu}_1^2}{\hat{\mu}_2 - \hat{\mu}_1^2} 
        \end{equation*}
        The sampling distribution can be hard to find. We will have to use boostrapping to do this. 
    \end{itemize}
\end{example}

\begin{definition}{\textbf{(Bootstrap)}}
    We can samping \emph{with replacement} of the data, and we calculate the parameter via any mean. The distribution of the parameter is the approximation of the sampling distribution. This method is called bootstrap.
\end{definition}

\begin{definition}{\textbf{(Consistent)}}
    Let $\hat{\theta}_n$ be an estimate of parameter $\theta$ based on sample of size $n$. Then $\hat{\theta}_n$ is said to be consistent in probability if $\hat{\theta}_n$ converges in probability to $\theta$ as $n$ appraoches infinity, that is for any $\varepsilon>0$:
    \begin{equation*}
        \mathbb{P}\bracka{\abs{\hat{\theta}_n - \theta} > \varepsilon}\rightarrow0 \quad \text{ as } \quad n \rightarrow\infty
    \end{equation*}
    The weak law of large number implies the sample moment converge in probability to population moment. 
\end{definition}

\subsection{Maximum Likelihood}

\begin{definition}{\textbf{(Maximum Likelihood)}}
    We will assume the data $X_i$ to be iid, and so the log-likelihood:
    \begin{equation*}
        l(\theta) = \log \prod^n_{i=1}p(X_i|\theta) = \sum^n_{i=1}\log f(X_i | \theta)
    \end{equation*}
\end{definition}


\begin{example}
    We will consider difference distributions and its maximum likelihood estimate:
    \begin{itemize}
        \item \textbf{Poisson Distribution:} The log-likelihood of the Poisson distribution is:
        \begin{equation*}
        \begin{aligned}
            l(\lambda) &= \sum^n_{i=1}(X_i \log \lambda - \lambda - \log X_i!) = \log \lambda\sum^n_{i=1}X_i - n\lambda - \sum^n_{i=1}\log X_i!
        \end{aligned}
        \end{equation*}
        We can see that its derivative is given as:
        \begin{equation*}
            l'(\lambda) = \frac{1}{\lambda} \sum^n_{i=1}X_i -n = 0
        \end{equation*}
        The MLE is equal to $\hat{\lambda} = \bar{X}$
        \item \textbf{Normal Distribution:} The log-likelihood is given by 
        \begin{equation*}
        \begin{aligned}
            l(\mu, \sigma^2) &= \log \prod^n_{i=1}\frac{1}{\sqrt{2\pi\sigma}} \exp\bracka{-\frac{(x_i-\mu)^2}{2\sigma^2}}  \\
            &= -n\log\sigma - \frac{n}{2}\log2\pi - \frac{1}{2\sigma^2}\sum^n_{i=1}(X_i-\mu)^2
        \end{aligned}
        \end{equation*}
        This leads to the following derivative:
        \begin{equation*}
            \frac{\partial l}{\partial\mu} = \frac{1}{\sigma^2}\sum^n_{i=1}(X_i-\mu) \qquad \frac{\partial l}{\partial\sigma} = -\frac{n}{\sigma}+ \sigma^{-3}\sum^n_{i=1}(X_i - \mu)^2
        \end{equation*}
        Setting the derivative to zero, and we have $\hat{\mu} = \bar{X}$ and we substitute the MLE for $\mu$ for $\sigma$ as we have $\hat{\sigma} = \sqrt{\frac{1}{n}\sum^n_{i=1}(X_i-\bar{X})^2}$. The sampling distribution is the same as method of moment. 
        \item \textbf{Gamma Distribution:} The log-likelihood is given by:
        \begin{equation*}
        \begin{aligned}
            l(\alpha,\lambda) &= \log\prod^n_{i=1}\frac{1}{\Gamma(\alpha)}\lambda^\alpha X_i^{\alpha-1}\exp(-\lambda X_i) \\
            &= n\alpha\log\lambda + (\alpha-1)\sum^n_{i=1}\log X_i - \lambda\sum^n_{i=1}X_i - \lambda\sum^n_{i=1}X_i - n\log\Gamma(\alpha)
        \end{aligned}
        \end{equation*}
        for $0\le x<\infty$. Now, we have the following derivative:
        \begin{equation*}
            \frac{\partial l}{\partial \alpha} = n\log \lambda + \sum^n_{i=1}\log X_i - n\frac{\Gamma'(\alpha)}{\Gamma} \qquad \frac{\partial l}{\partial\lambda} = \frac{n\alpha}{\lambda} - \sum^n_{i=1}X_i
        \end{equation*}
        Setting the second partial to zero as $\hat{\lambda} = (n\hat{\alpha})/(\sum^n_{i=1}X_i) = \hat{\alpha}/\bar{X}$. Now $\alpha$ can be solved by non-linear equation via iterative method:
        \begin{equation*}
            n\log\hat{\alpha} - n\log\bar{X} + \sum^n_{i=1}\log X_i - n\frac{\Gamma'(\hat{\alpha})}{\Gamma(\hat{\alpha})} = 0
        \end{equation*}
        The sampling distribution can be found by bootstrapping. 
        \item \textbf{Multinomial-Cell Distribution:} We have the following log-likelihood to be:
        \begin{equation*}
            l(p_1,\dots,p_m) = \log \frac{n!}{\prod^m_{i=1} X_i!}\prod^m_{i=1}p_i^{X_i} = \log n! - \sum^m_{i=1}\log X_i! + \sum^m_{i=1}x_i\log p_i
        \end{equation*}
        Maximizing the likelihood would be subject to contraint as we have have the following Lagragian:
        \begin{equation*}
            \mathcal{L}(p_1,\dots,p_m) = \log n! - \sum^m_{i=1}\log x_i! + \sum^m_{i=1}x_i\log p_i + \lambda\bracka{\sum^m_{i=1}p_i - 1}
        \end{equation*}
        Setting the partial derivative to be equal to zero: 
        \begin{itemize}
            \item As we have the following system of equation: $\hat{p}_j = -x_j/\lambda $ for $j=1,\dots,m$ summing both equation as we have: $1 = -n/\lambda$ or $\lambda =-n$ and so $\hat{p}_j = x_j/n$. 
            \item The sampling distribution of $\hat{p}_j$ is determined by the distribution of $x_j$, which is biomial.
        \end{itemize}
    \end{itemize}
\end{example}

\begin{theorem}
    Under appropriate smoothness conditions on $f$, the MLE from an iid sample is consistent.
\end{theorem}
\begin{proof}
    Consider maximizing the following values, given the $X_1,X_2,\dots,X_n \sim p(X|\theta_0)$:
    \begin{equation*}
        \frac{1}{n}l(\theta) = \frac{1}{n}\sum^n_{i=1}\log p(X_i | \theta)
    \end{equation*}
    as $n$ tends to infinity, the law of large number implies that:
    \begin{equation*}
    \begin{aligned}
        \frac{1}{n}l(\theta) &\rightarrow \mathbb{E}_{X \sim p(X | \theta_0)}[\log p(X | \theta)] \\
        &= \int p(x|\theta_0)\log p(x|\theta) \dby x
    \end{aligned}
    \end{equation*}
    The $\theta$ that maximizes $l(\theta)$ should be closed to the $\theta$ that maximizes $\mathbb{E}[\log f(X|\theta)]$ (again not shown). We consider the derivative:
    \begin{equation*}
        \frac{\partial}{\partial \theta}\int p(x|\theta_0)\log p(x|\theta)\dby x = \int p(x|\theta)\frac{p(x|\theta_0)}{p(x|\theta)} \cfrac{\partial}{\partial \theta} \dby x
    \end{equation*}
    If $\theta=\theta_0$, this equation becomes:
    \begin{equation*}
        \int \frac{\partial}{\partial\theta}p(x|\theta_0)\dby x = \frac{\partial}{\partial \theta} \int p(x|\theta_0)\dby x = \frac{\partial}{\partial \theta}(1) = 0
    \end{equation*}
    This shows that $\theta_0$ is stationary and (hopefully) it is a maximum. The assumption of smoothness on $f$ must be strong enough to justify this. 
\end{proof}

\begin{lemma}
    Define $I(\theta)$ by:
    \begin{equation*}
        I(\theta) = \mathbb{E}\brackb{\frac{\partial}{\partial \theta} \log p(X|\theta)}^2 = - \mathbb{E}\brackb{\frac{\partial^2}{\partial\theta^2}\log p(X|\theta)}
    \end{equation*}
    Under appropriate smoothness conditions on $p$, this can be expressed on the right-hand side.
\end{lemma}
\begin{proof}
    Observe that $\int p(x|\theta)\dby x = 1$, and so we have, the following observation:
    \begin{equation*}
        \frac{\partial}{\partial\theta} \int p(X|\theta)\dby x = 0
        \qquad \frac{\partial}{\partial\theta}p(x|\theta) = p(x|\theta)\brackb{\frac{\partial}{\partial\theta}\log p(x|\theta)}
    \end{equation*}
    Combinding this with identity, as we have (take the second derivative to be):
    \begin{equation*}
    \begin{aligned}
        0 = \frac{\partial}{\partial\theta}\int p(x|\theta)\dby x &= \int \brackb{\frac{\partial}{\partial \theta}\log p(x|\theta)}p(x|\theta)\dby x \\
        &= \int \brackb{\frac{\partial^2}{\partial\theta^2}\log p(x|\theta)}p(x|\theta)\dby x + \int \brackb{\frac{\partial}{\partial \theta}\log p(x|\theta)}^2p(x|\theta)\dby x \\
    \end{aligned}
    \end{equation*}
    And so we have the lemma is proven.
\end{proof}

\begin{theorem}
    Under smoothness condition on $f$, the probability distribution of $\sqrt{nI(\theta_0)}(\hat{\theta} - \theta_0)$ thends to a standard normal distribution
\end{theorem}
\begin{proof}
    The following is sketch of proof. Consider the Taylor series expansion (of $l'(\hat{\theta})$), as we have:
    \begin{equation*}
    \begin{aligned}
        0 = l'(\hat{\theta}) &\approx l'(\theta_0) + (\hat{\theta} - \theta_0)l''(\theta) \\
        (\hat{\theta} - \theta_0)&\approx \frac{-l'(\theta_0)}{l''(\theta_0)} \\
        \sqrt{n}(\hat{\theta} - \theta_0) &\approx\frac{-{n}^{-1/2}l'(\theta_0)}{n^{-1}l''(\theta_0)}
    \end{aligned}
    \end{equation*}
    We consider the numeraor of this last expression. Its expectation is given as:
    \begin{equation*}
    \begin{aligned}
        \mathbb{E}\brackb{n^{-1/2}l'(\theta_0)} = n^{-1/2}\sum^n_{i=1}\mathbb{E}\brackb{\frac{\partial}{\partial\theta}\log p(X_i | \theta_0)} = 0
    \end{aligned}
    \end{equation*}
    As we have $\theta_0$, which is the fixed point (see thoerem above). Now, consider the variance of the quantity:
    \begin{equation*}
        \operatorname{var}\brackb{n^{-1/2} l'(\theta_0)} = \frac{1}{n}\sum^n_{i=1}\mathbb{E}\brackb{\frac{\partial^2}{\partial \theta^2} \log p(x_i | \theta_0) }^2 = I(\theta_0)
    \end{equation*}
    Consider the denominator to be. Together with the law of large number, the expression converges to:
    \begin{equation*}
        \frac{1}{n} l''(\theta_0) = \frac{1}{n}\sum^n_{i=1}\frac{\partial^2}{\partial\theta^2} \log p(x_i|\theta_0) \longrightarrow \mathbb{E}\brackb{\frac{\partial^2}{\partial\theta^2}\log p(x|\theta_0)} = -I(\theta_0)
    \end{equation*}
    Thus, we have:
    \begin{equation*}
        n^{1/2}(\hat{\theta} - \theta_0) \approx \frac{n^{-1/2}l'(\theta_0)}{I(\theta_0)}
    \end{equation*}
    We have the following mean and variance of the ratio to be:
    \begin{equation*}
    \begin{aligned}
        &\mathbb{E}[n^{1/2}(\hat{\theta} - \theta_0)] \approx 0 \\
        &\operatorname{var}[n^{1/2}(\hat{\theta} - \theta_0)] \approx \frac{I(\theta_0)}{I^2(\theta_0)} = \frac{1}{I(\theta_0)}
    \end{aligned}
    \end{equation*}
    And so we have $\operatorname{var}(\hat{\theta} - \theta_0)\approx 1/(nI(\theta_0))$. 
    % The CLT may be applied to $l'(\theta_0)$, which is the sum of iid random variables:
    % \begin{equation*}
    %     l'(\theta_0) =\sum^n_{i=1}\frac{\partial}{\partial\theta}\log f(x_i | \theta_0)
    % \end{equation*}
    Thus the equation is proven.
\end{proof}

\begin{remark}
    For an iid sample, the MLE is the maximizer of the log-likelihood function $l(\theta) = \sum^n_{i=1}\log p(X_i|\theta)$ has the asymptotic variance that is given as:
    \begin{equation*}
        \frac{1}{nI(\theta_0)} = -\frac{1}{\mathbb{E}[l''(\theta_0)]}
    \end{equation*}
    When $\mathbb{E}[l''(\theta_0)]$ is large, meaning that $l(\theta)$ is changing very rapidly in a vincinity of $\theta_0$ and the variance of the maximizer is small. 
\end{remark}

\begin{remark}{\textbf{(Confidence Interval for Mean and Variance Estimate)}}
    Consider the maximum likelihood estimate of $\mu$ and $\sigma^2$ from an iid normal sample to be:
    \begin{equation*}
        \hat{\mu} = \bar{X} \qquad \hat{\sigma}^2 = \frac{1}{n}\sum^n_{i=1}(X_i - \bar{X})^2
    \end{equation*}
    There are various confidence interval on each of the likelihood estimation as we have:
    \begin{itemize}
        \item Confidence interval of $\mu$ is based on:
        \begin{equation*}
            \frac{\sqrt{n}(\bar{X} - \mu)}{S} \sim t_{n-1} \qquad \text{ where } \qquad S^2 = \frac{1}{n-1}\sum^n_{i=1}(X_i-\bar{X})^2
        \end{equation*}
        Let $t_{n-1}(\alpha/2)$ denote the point beyound which $t$ distribution with $n-1$ degree of freedom has probability $\alpha/2$, to be:
        \begin{equation*}
            \mathbb{P}\bracka{-t_{n-1}(\alpha/2)\le \frac{\sqrt{n}(\bar{X} - \mu)}{S} \le t_{n-1}(\alpha/2)} = 1-\alpha
        \end{equation*}
        The inequality can be manipulated to yields:
        \begin{equation*}
            \mathbb{P}\bracka{\bar{X} - \frac{S}{\sqrt{n}} t_{n-1}(\alpha/2) \le \mu \le \bar{X} + \frac{S}{\sqrt{n}}t_{n-1}(\alpha/2) } = 1-\alpha
        \end{equation*}
        The probability that $\mu$ lies in the interval is $1-\alpha$. 
        \item Let's consider the conditional interval $\sigma^2$, as we have the following distribution:
        \begin{equation*}
            \frac{n\hat{\sigma}^2}{\sigma^2} \sim \chi^2_{n-1}
        \end{equation*}
        Let $\chi^2_m(\alpha)$ denote the point beyound which the chi-square distribution with $m$ degree of freedom that has probability $\alpha$:
        \begin{equation*}
            \mathbb{P}\bracka{\chi^2_{n-1}(1-\alpha/2)\le \frac{n\hat{\sigma}^2}{\sigma^2} \le \chi^2_{n-1}(\alpha/2)} = 1-\alpha
        \end{equation*}
        Manipulation of the inequality yields:
        \begin{equation*}
            \mathbb{P}\bracka{\frac{n\hat{\sigma}^2}{\chi^2_{n-1}(\alpha/2)} \le \sigma^2 \le \frac{n\hat{\sigma}^2}{\chi^2(1-\alpha/2)}} = 1-\alpha
        \end{equation*}
        \item For a general maximum likelihood methods, one can consider the distribution of $\sqrt{nI(\hat{\theta})}(\hat{\theta} - \theta_0)$, where it is normally distributed, and so we have the following intervales:
        \begin{equation*}
            \mathbb{P}\bracka{-z(\alpha/2) \le \sqrt{nI(\hat{\theta})}(\hat{\theta}-\theta_0) \le z(\alpha/2) } \approx 1-\alpha
        \end{equation*}
        which we can yields the confidence interval, as we have:
        \begin{equation*}
            \mathbb{P}\bracka{ -\frac{z(\alpha/2)}{\sqrt{nI(\hat{\theta})}} \le \theta_0 \le \frac{z(\alpha/2)}{\sqrt{nI(\hat{\theta})}}  } 
        \end{equation*}
        \item For the estimation for \emph{random multinomial}. The counts are not iid, so the variance of the parameter estimate is of the form $1/[nI(\theta)]$ can't be used. It can be shown that:
        \begin{equation*}
            \operatorname{var}(\hat{\theta}) \approx \frac{1}{\mathbb{E}[l'(\theta_0)^2]} = -\frac{1}{\mathbb{E}[l''(\theta_0)]}
        \end{equation*}
        Please note that this is used to construct the confidence interval instead of above. 
    \end{itemize}
\end{remark}

\subsection{Cramer-Rao Lower Bound}

 \begin{definition}{\textbf{(Efficiency of Estimates)}}
    Given $2$ estimates $\hat{\theta}$ and $\tilde{\theta}$ of a parameter $\theta$, the efficiency of $\hat{\theta}$ and $\tilde{\theta}$ is defined to be:
    \begin{equation*}
        \operatorname{eff}(\hat{\theta}, \tilde{\theta}) = \frac{\operatorname{var}(\tilde{\theta})}{\operatorname{var}(\hat{\theta})}
    \end{equation*}
 \end{definition}

 \begin{theorem}
    Let $X_1,\dots,X_n$ be iid with density function $p(x|\theta)$. Let $T = t(X_1,\dots,X_n)$ be unbiased estimate of $\theta$. Then under smoothness assumption on $p(x|\theta)$, we have:
    \begin{equation*}
        \operatorname{var}(T) \ge \frac{1}{nI(\theta)}
    \end{equation*}
 \end{theorem}
 \begin{proof}
    Let the following value:
    \begin{equation*}
    \begin{aligned}
        Z &= \sum^n_{i=1}\frac{\partial}{\partial\theta}\log p(X_i|\theta) = \sum^n_{i=1}\frac{1}{p(X_i|\theta)} \frac{\partial}{\partial\theta } p(X_i|\theta)
    \end{aligned}
    \end{equation*}
    We already show that $\mathbb{E}[Z] = 0$. Because the correlation coefficient of $Z$ and $T$ is less than or equal to $1$ in absolute value as:
    \begin{equation*}
        \operatorname{cov}^2(Z, T) \le \operatorname{var}(Z)\operatorname{var}(T)
    \end{equation*}
    Furthermore, we have shown that (from the lemma of $I(\theta)$):
    \begin{equation*}
        \operatorname{var}\brackb{\frac{\partial}{\partial\theta} \log p(X|\theta)} = I(\theta)
    \end{equation*}
    and so $\operatorname{var}(Z) = nI(\theta)$. The proof will be complete by showing that $\operatorname{cov}(Z, T) = 1$. Please note that (follows product rule):
    \begin{equation*}
        \bracka{\sum^n_{i=1}\frac{1}{p(X_i|\theta)} \frac{\partial}{\partial\theta } p(X_i|\theta)}\bracka{\prod^n_{j=1}f(x_j | \theta)} = \frac{\partial}{\partial\theta}\prod^n_{i=1}f(x_i|\theta)
    \end{equation*}
    Since $Z$ has mean of $0$, we have:
    \begin{equation*}
    \begin{aligned}
        \operatorname{cov}(Z, T) &= \mathbb{E}[ZT] \\
        &= \int\cdots\int t(x_1,\dots,x_n) \brackb{\sum^n_{i=1}\frac{1}{p(X_i|\theta)} \frac{\partial}{\partial\theta } p(X_i|\theta)}\prod^n_{j=1}f(x_j|\theta)\dby x_j \\
        &= \int\cdots\int t(x_1,\dots,x_n)\frac{\partial}{\partial\theta}\prod^n_{i=1}f(x_i|\theta) \dby x_i \\
        &= \frac{\partial}{\partial\theta}\int\cdots\int t(x_1,\dots,x_n)\prod^n_{i=1}f(x_i|\theta) \dby x_i \\
        &= \frac{\partial}{\partial\theta} \mathbb{E}[T] = \frac{\partial}{\partial\theta} \theta = 1
    \end{aligned}
    \end{equation*}
    This proves the inequality as we have. 
\end{proof}

\begin{definition}{\textbf{(Efficient)}}
    The unbiased estimate whose variance achieves this lower bound is said to be efficient. Since the asymptotic variance of maximum likelihood estimate is equal to lower bound, it is said to be asymptotically efficient. 
\end{definition}

\subsection{Sufficient Statistics}

\begin{definition}
    A statistics $T(X_1,\dots,X_n)$ is said to be sufficient for $\theta$ if conditional distribution of $X_1,\dots,X_n$ given $T=t$ doesn't depends on $\theta$ or any value of $t$. 
\end{definition}

\begin{theorem}
    A necessary and sufficient condition for $T(X_1,\dots,X_n)$ to be sufficient for a parameter $\theta$ is the joint probability function factors in the form of:
    \begin{equation*}
        p(x_1,\dots,x_n|\theta) = g[T(x_1,\dots,x_n), \theta]h(x_1,\dots,x_n)
    \end{equation*}
\end{theorem}
\begin{proof}
    We will consider it to be in discrete case. Suppsoe that the frequency function factors. To simplify notation, we let $\boldsymbol X$ denotes $(X_1,\dots,X_n)$ and $\boldsymbol x$ denotes $(x_1,\dots,x_n)$. We have:
    \begin{equation*}
    \begin{aligned}
        P(T = t) &= \sum_{T(x) = t}P(\boldsymbol X=\boldsymbol x) \\
        &= g(t, \theta) \sum_{T(x) = t} h(\boldsymbol x)
    \end{aligned}
    \end{equation*}
    We then have:
    \begin{equation*}
        P(\boldsymbol X=\boldsymbol x | T = t) = \frac{P(\boldsymbol X = \boldsymbol x, T = t)}{P(T = t)} = \frac{h(\boldsymbol x)}{\sum_{T(\boldsymbol X) = t}h(\boldsymbol x)}
    \end{equation*}
    This conditional distributed doesn't depend on $\theta$. To show that the conclusion holds in other direction, suppose that the conditional distribution of $\boldsymbol X$ given $T$ is independent of $\theta$. Let:
    \begin{equation*}
        g(t, \theta) = P(T=t|\theta) \qquad h(\boldsymbol x) = P(\boldsymbol X=\boldsymbol x | T = t)
    \end{equation*}
    We then have:
    \begin{equation*}
    \begin{aligned}
        P(\boldsymbol X = \boldsymbol x | \theta) &= P(T = t | \theta)P(\boldsymbol X=\boldsymbol x |T=t) \\
        &= g(t, \theta)h(\boldsymbol x)
    \end{aligned}
    \end{equation*}
\end{proof}

\begin{corollary}
    If $T$ is sufficient for $\theta$, the MLE is a function of $T$.
\end{corollary}
\begin{proof}
    The likelihood is $g(T, \theta)h(\boldsymbol x)$, which depends on $\theta$ only through $T$. To maximize this quantity, we need to maximize $g(T,\theta)$
\end{proof}

\begin{theorem}{\textbf{(Rao-Blackwell Theorem)}}
    Let $\hat{\theta}$ be an estimator of $\theta$ with $\mathbb{E}[\hat{\theta}^2]<\infty$ for all $\theta$. Suppose that $T$ is sufficient statistics for $\theta$, and let $\tilde{\theta} = \mathbb{E}[\hat{\theta} | T]$, then for all $\theta$:
    \begin{equation*}
        \mathbb{E}[\tilde{\theta} - \theta]^2 \le \mathbb{E}[\hat{\theta} - \theta]^2
    \end{equation*}
    The inequality is strict unless $\hat{\theta} = \tilde{\theta}$.
\end{theorem}
\begin{proof}
    First note that from the property of iterated condition expectation, we have:
    \begin{equation*}
        \mathbb{E}[\tilde{\theta}] = \mathbb{E}[\mathbb{E}[\hat{\theta} | T]] = \mathbb{E}[\tilde{\theta}]
    \end{equation*}
    To compare the square-error, we will have to only consider their varince:
    \begin{equation*}
    \begin{aligned}
        &\operatorname{var}(\hat{\theta}) = \operatorname{var}[\mathbb{E}[\hat{\theta} | T]] + \mathbb{E}[\operatorname{var}[\hat{\theta} | T]] \\
        &= \operatorname{var}(\tilde{\theta}) + \mathbb{E}[\operatorname{var}(\hat{\theta} | T)]
    \end{aligned}
    \end{equation*}
    Thus $\operatorname{var}(\hat{\theta}) > \operatorname{var}(\tilde{\theta})$ unless $\operatorname{var}(\hat{\theta} | T) = 0$, which is when $\hat{\theta}$ is a function of $T$, which implies $\hat{\theta} =\tilde{\theta}$.
\end{proof}


\section{Testing Hypothesis and Goodness of Fit}

\subsection{Introduction}

\begin{definition}{\textbf{(Likelihood Ratio)}}
    Consider the two hypothesis to be $H_0$ and $H_1$, we have the following, posterior:
    \begin{equation*}
        P(H_0 | x) = \frac{P(x|H_0)P(H_0)}{P(x)} \qquad P(H_1 | x) = \frac{P(x|H_1)P(H_1)}{P(x)} 
    \end{equation*}    
    The ratio is given as:
    \begin{equation*}
        \frac{P(H_0|x)}{P(H_1|x)} = \frac{P(H_0)}{P(H_1)}\frac{P(x|H_0)}{P(x|H_1)}
    \end{equation*}
    This is the product of the ratio of prior probability and the likelihood ratio. Now, we would like to choose the hypothesis $H_0$ if, we have:
    \begin{equation*}
        \frac{P(H_0|x)}{P(H_1|x)} = \frac{P(H_0)}{P(H_1)}\frac{P(x|H_0)}{P(x|H_1)} > 1 \quad \iff \quad 
        \frac{P(x|H_0)}{P(x|H_1)} > c
    \end{equation*}
    where value of $c$ depends upon your prior probability. 
\end{definition}

\begin{definition}{\textbf{(Neyman-Pearson Paradigm)}}
    One hypothesis is singled out as \emph{null hypothesis} $H_0$ and other as \emph{alternative hypothesis} $H_1$. We have the following terminology as:
    \begin{itemize}
        \item Rejecting $H_0$ when it is true is called \emph{type I error}.
        \item Probability of a type I error is called \emph{significance level} and it is denoted as $\alpha$.
        \item Accepting the null hypothesis when it is false is called \emph{type II error}, and it is denoted by $\beta$. 
        \item The probability that the null hypothesis is rejected when it is false is called \emph{power} of the test, which is equal to $1-\beta$.
        \item The likelihood ratio is called the \emph{test statistics}.
        \item Set of values of the test statistics that leads to rejection of the null hypothesis is called \emph{rejection region}, and set of values that leads to acceptance is called \emph{acceptance region}
        \item The probability distribution of test statistics when the null hypothesis is true is called \emph{null distribution}.
    \end{itemize}
\end{definition}

\begin{definition}{\textbf{(Simple Hypothesis)}}
    If the null and alternative hypothesis each completely specify the probability distribution. This kind of setting is called simple hypothesis. 
\end{definition}

\begin{lemma}{\textbf{(Neyman-Peason)}}
    Suppose that $H_0$ and $H_1$ are simple hypothesis:
    \begin{itemize}
        \item The test that rejects $H_0$ whenever the likelihood ratio is less than $c$ and significance level $\alpha$. 
        \item Then any other test for which significance level is less than or equal to $\alpha$ has power less than or equal to that of the likelihood ratio test. 
    \end{itemize} 
\end{lemma}
\begin{proof}
    Let $p(x)$ denote the pdf or frequency function of the observation. 
    \begin{itemize}
        \item A test of $H_0 : p(x) = p_0(x)$ and $H_1 : p(x) = p_1(x)$ amounts to using a decision function:
        \begin{equation*}
            d(x) = \begin{cases}
                0 & \text{ if } H_0 \text{ is accepted} \\
                1 & \text{ if } H_1 \text{ is rejected} \\
            \end{cases}
        \end{equation*}
        \item Since $d(X)$ is a Bernoulli random varaible, where we have:
        \begin{itemize}
            \item Significance Level: $\mathbb{E}_0[d(X)] = P_0(d(X) = 1)$
            \item Power: $\mathbb{E}_1[d(X)] = P_1(d(X) = 0)$
        \end{itemize}
        \item If we consider the likelihood ratio test as the decision function:
        \begin{equation*}
            d(x) = \begin{cases}
                1 & \text{ if } p_0(X) < cp_1(X) \\
                0 & \text{ otherwise } \\
            \end{cases}
        \end{equation*}
        Please note that $\mathbb{E}_{X\sim p_0(X)}[X] = \alpha$. 
        \item Let $d^*(X)$ be the decision function of another test satisfying $\mathbb{E}_0[d^*(X)] \le \mathbb{E}_0[d^*(x)] = \alpha$. 
        \item Consider the following inequalities:
        \begin{equation*}
            d^*(x)[cp_1(x) - p_0(x)] \le d(x)[cp_1(x) - p_0]
        \end{equation*}
        This follows from the $d(x) = 1$, where $cf_1(x) - f_0(x)> 0$ and if $d(x) = 0$. where $cf_1(x) - f_0(x)\le0$
        \item Integrating the both sides of the inequality above with respected to $x$ as:
        \begin{equation*}
            c \mathbb{E}_1[d^*(X)] - \mathbb{E}_0[d^*(X)] \le c \mathbb{E}_1[d(X)] - \mathbb{E}_0[d(X)]
        \end{equation*}
        and, so we have:
        \begin{equation*}
            \mathbb{E}_0[d(X)] - \mathbb{E}_0[d^*(X)] \le c\brackb{\mathbb{E}_1[d(X)] - \mathbb{E}_1[d^*(X)]}
        \end{equation*}
        Since the LHS of this inequality is non-negative, we have: $\mathbb{E}[d^*(X)] \le \mathbb{E}_A[d(X)]$
    \end{itemize}
\end{proof}

\begin{example}{\textbf{(First Test)}}
    Consider $X_1,\dots,X_n$ be random sample from normal distribution, with unknown mean and variance $\sigma^2$. Given $2$ hypothesis:
    \begin{equation*}
        H_0 : \mu = \mu_0 \qquad H_1 : \mu = \mu_1
    \end{equation*}
    where $\mu_1$ and $\mu_0$ are constant. Consider a significance level of $\alpha$. Then consider likelihood ratio:
    \begin{equation*}
        \frac{f_0(\boldsymbol X)}{f_1(\boldsymbol X)} = \cfrac{\exp\brackb{-\cfrac{1}{2\sigma^2} \sum^n_{i=1}(X_i-\mu_0)^2 }}{\exp\brackb{-\cfrac{1}{2\sigma^2} \sum^n_{i=1}(X_i-\mu_1)^2 }}
    \end{equation*}
    To consider the ratio, we consider the value of $\sum^n_{i=1}(X_i - \mu_1)^2 - \sum^n_{i=1}(X_i - \mu_0)^2$. Expanding the squares:
    \begin{equation*}
        2n\bar{X}(\mu_0 -\mu_1) + n\mu_1^2 - n\mu_0^2
    \end{equation*}
    There are $2$ conditions, so that the likelihood is small:
    \begin{itemize}
        \item If $\mu_0-\mu_1>0$, the likelihood ratio is small if $\bar{X}$ is small. 
        \item If $\mu_0 - \mu_1 < 0$, the likelihood ratio is small if $\bar{X}$ is large. 
    \end{itemize}
    Let's consider the later case. Likelihood-ratio rejects for $\bar{X} > x_0$ for some $x_0$, which we will choose it to give a test of desired level $\alpha$. This means choosing $\mathbb{P}(\bar{X} > x_0) = \alpha$ if $H_0$ is true:
    \begin{equation*}
        \mathbb{P}(\bar{X}>x_0) = \mathbb{P}\bracka{ \frac{\bar{X} - \mu_0}{\sigma/\sqrt{n}} > \frac{x_0 - \mu_0}{\sigma/\sqrt{n}}}
    \end{equation*}
    The null distribution of $\bar{X}$ is a normal distribution with mean $\mu_0$ and variance $\sigma^2/n$, then, we can solve:
    \begin{equation*}
        \frac{x_0 - \mu_0}{\sigma/\sqrt{n}} = z(\alpha)
    \end{equation*}
    for $x_0$ in order to find the rejection region for level $\alpha$ test.
\end{example}

\begin{definition}{\textbf{(P-Value)}}
    As we can see, the testing requires only the null distribution, and we are required to consider the significance level $\alpha$ (which should be $0.01$ and $0.05$). P-value is the smallest significance level at which the null hypothesis would be rejected. 
\end{definition}

\subsection{More Complex Hypothesis Testing}

\begin{definition}{\textbf{(Uniformly Most Powerful)}}
    If the alternative hypothesis $H_1$ is composite, a test is most powerful for every simple alternative in $H_1$ is said to be uniformly most powerful. 
\end{definition}

\begin{example}{\textbf{(2-Sided Test)}}
    Consider $X_1,\dots,X_n$ be random sample from normal distribution, with unknown mean and variance $\sigma^2$. Given $2$ hypothesis:
    \begin{equation*}
        H_0 : \mu = \mu_0 \qquad H_1 : \mu \ne \mu_0
    \end{equation*}
    Please note that in this example, this kind of hypothesis is called two-sided alternative. Consider the test at a specific level $\alpha$ that reject for $\abs{\bar{X} - \mu_0} > x_0$, where $x_0$ is determined such that $\mathbb{P}(\abs{\bar{X} - \mu_0} > x_0) =\alpha$ if $H_0$ is true. We can see that $x_0 = z(\alpha/2)\sigma/\sqrt{n}$:
    \begin{equation*}
    \begin{aligned}
        &\abs{\bar{X}-\mu_0} < \frac{z(\alpha/2)\sigma}{\sqrt{n}}\\
        \iff& \bar{X} - \frac{z(\alpha/2)\sigma}{\sqrt{n}} \le \mu_0 < \bar{X} + \frac{z(\alpha/2)\sigma}{\sqrt{n}} \\
    \end{aligned}
    \end{equation*}
    A $100(1-\alpha)\%$ interval for $\mu$ is give, and so if $\mu_0$ is in the interval, then we accept the null hypothesis. 
\end{example}

\begin{theorem}
    Suppose that for every value $\theta_0$ in $\Theta$ there is a test at level $\alpha$ of the hypothesis $H_0 : \theta = \theta_0$. Denote the acceptance region of the test by $A(\theta_0)$. Then set:
    \begin{equation*}
        C(\boldsymbol X) = \brackc{\theta :\boldsymbol X \in A(\theta)}
    \end{equation*}
    is a $100(1-\alpha)\%$ confidence region for $\theta$. 
\end{theorem}
\begin{remark}
    This means that a $100(1-\alpha)\%$ confidence region for $\theta$ consists of all those values of $\theta_0$ for which the hypothesis that $\theta$ equals $\theta_0$ will not be rejected at level $\alpha$.
\end{remark}
\begin{proof}
    Because $A$ is the acceptance region of a test at level $\alpha$:
    \begin{equation*}
        \mathbb{P}[\boldsymbol X \in A(\theta_0) | \theta = \theta_0] = 1-\alpha 
    \end{equation*}
    Now, we have:
    \begin{equation*}
        \mathbb{P}[\theta_0 \in C(\boldsymbol X) | \theta = \theta_0] = \mathbb{P}[\boldsymbol X \in A(\theta_0) | \theta=\theta_0] = 1-\alpha
    \end{equation*}
    by the definition of $C(\boldsymbol X)$
\end{proof}

\begin{definition}{\textbf{(Generalized Likelihood Ratio Test)}}
    Suppose that the observation: $\boldsymbol X = (X_1,\dots,X_n)$  have a joint density $p(\boldsymbol x | \theta)$:
    \begin{itemize}
        \item Then $H_0$ may specify that $\theta \in \omega_0$ where $\omega_0$ is subset of all possible values of $\theta$
        \item For $H_1$ we consider $\omega_1$ is disjoint from $\omega_0$. 
    \end{itemize}
    Let $\Omega = \omega_0 \cup \omega_1$. The generalized likelihood ratio is $\Lambda^*$ or with the truncated version $\Lambda$ as the small value of $\Lambda^*$ tends to discredit $H_0$: 
    \begin{equation*}
        \Lambda^* = \frac{\max_{\theta\in \omega_0} l(\theta)}{\max_{\theta \in \omega_1}l(\theta)}
        \qquad\Lambda = \frac{\max_{\theta \in \omega_0}l(\theta)}{\max_{\theta \in \Omega}l(\theta)}
    \end{equation*} 
    Note that $\Lambda = \min(\Lambda^*, 1)$. The rejection region is given as $\Lambda \le \lambda_0$, where the threshold $\lambda_0$ is choosen so that 
    \begin{equation*}
        \mathbb{P}(\Lambda \le \lambda_0 | H_0) = \alpha
    \end{equation*}
\end{definition}

\begin{example}{\textbf{(Testing Normal Mean)}}
    Consider $X_1,\dots,X_n$ be random sample from normal distribution, with unknown mean and variance $\sigma^2$. Given $2$ hypothesis:
    \begin{equation*}
        H_0 : \mu = \mu_0 \qquad H_1 : \mu \ne \mu_0
    \end{equation*}
    We have the following specification:
    \begin{equation*}
        \omega_0 = \brackc{\mu_0} \qquad \omega_1 = \brackc{\mu|\mu\ne\mu_0} \qquad \Omega = \brackc{-\infty < \mu < \infty}
    \end{equation*}
    If we maximize over $\omega_0$, as it has only one point, the numerator. For the denominator, we it is clear that the MLE is $\bar{X}$ and so:
    \begin{equation*}
        \max_{\theta \omega_1}l(\theta) = \frac{1}{(2\sigma\pi)^n}\exp\bracka{-\frac{1}{2\sigma^2}\sum^n_{i=1}(X_i - \mu_0)^2} \qquad 
        \max_{\theta \Omega}l(\theta) = \frac{1}{(2\sigma\pi)^n}\exp\bracka{-\frac{1}{2\sigma^2}\sum^n_{i=1}(X_i - \bar{X})^2}
    \end{equation*}
    The ratio is given as:
    \begin{equation*}
    \begin{aligned}
        &\Lambda = \exp\bracka{-\frac{1}{2\sigma^2}\brackb{\sum^n_{i=1}(X_i - \mu_0)^2 - \sum^n_{i=1}(X_i - \bar{X})^2 }} \\
        \iff& \begin{aligned}[t]
            -2\log\Lambda &= \frac{1}{\sigma^2}\bracka{\sum^n_{i=1}(X_i-\mu_0)^2 - \sum^n_{i=1}(X_i-\bar{X})^2}\\
            &= \frac{n(\bar{X} - \mu_0)^2}{\sigma^2}
        \end{aligned}
    \end{aligned}
    \end{equation*}
    Rejecting for small value of $\Lambda$ is equivalent to reject the large value of $-2\log\Lambda$. Together with the identity that $\sum^n_{i=1}(X_i-\mu_0)^2 = \sum^n_{i=1}(X_i - \bar{X})^2 + n(\bar{X} - \mu_0)^2$ . It follows that, under $H_0$:
    \begin{itemize}
        \item $\bar{X} \sim \mathcal{N}(\mu_0, \sigma^2/n)$, which implies that $\sqrt{n}(\bar{X}-\mu_0)/\sigma\sim\mathcal{N}(0, 1)$ 
        \item $-2\log\Lambda\sim\chi^2_1$ is implied from above. 
    \end{itemize}
    We can now construct the rejection region for any significance level to be:
    \begin{equation*}
        \frac{n}{\sigma^2}(\bar{X} - \mu_0)^2 > \chi^2_1(\alpha)
    \end{equation*}
    where $P(Z > \chi^2_1(\alpha)) = \alpha$, recall that we are rejecting the large value of $-2\log \Lambda$. This links back to the original consideration as, we this inequality is equivalent to:
    \begin{equation*}
        \abs{\bar{X}-\mu_0} \ge \frac{\sigma}{\sqrt{n}}z(\alpha/2)
    \end{equation*}
\end{example}

\begin{theorem}
    Under smoothness condition on the probability density, the null distribution of $-2\log\Lambda$ tends to a chi-square distribution with degree of freedom of $\dim\Omega - \dim\omega_0$ as the sample size tends to infinity.
\end{theorem}

\begin{example}{\textbf{(Tests for Multinomial Distribution/Goodness of Fit)}}
    We consider the following testing scenario
    \begin{itemize}
        \item $H_0$: The cell probabilities $p = p(\theta)$ for $\theta \in \omega_0$ (maybe unknown, with dimension of $k$) is constrained on some way.
        \item $H_1$: Cell probabilities are free except the constriants such that they are non-negative and sum to $1$. 
    \end{itemize}
    We have $\Omega$ to be set of $m$ non-negative numbers that sum to one. We have:
    \begin{equation*}
        \max_{p\in\omega_0}\bracka{\frac{n!}{x_1!\cdots x_m!}}p_1(\theta)^{x_1}\cdots p_m(\theta)^{x_m}
    \end{equation*}
    where $x_i$ are observed counts in $m$ cells. We will denote $\hat{\theta}$ as the MLE of $\theta$. For the denominator, with unrestricted MLE, we have $\hat{p}_i = x_i/n$, and so, the ratio is:
    \begin{equation*}
    \begin{aligned}
        &\Lambda = \cfrac{\cfrac{n!}{x_1!\cdots x_m!}p_1(\hat{\theta})^{x_1}\cdot\hat{\theta})^{x_m}}{\cfrac{n!}{x_1!\cdots x_m!}\hat{p}^{x_1}_1\cdots\hat{p}^{x_m}_m} =\prod^m_{i=1}\bracka{\frac{p_i(\hat{\theta})}{\hat{p}_i}}^{x_i} \\
        \implies& \begin{aligned}[t]
            -2\log\Lambda &= -2n\sum^m_{i=1}\hat{p}_i\log\bracka{\frac{p_i(\hat{\theta})}{\hat{p}_i}}
            = 2\sum^m_{i=1}O_i\log\bracka{\frac{O_i}{E_i}}
        \end{aligned}
    \end{aligned}
    \end{equation*}
    As we have $x_i = n\hat{p}_i$, $O_i=n\hat{p}_i$ and $E_i=np_i(\hat{\theta})$. Let's consider the test statistics:
    \begin{itemize}
        \item $\Omega$ allows cell probability to be free (but have to be sum to $1$) so $\dim \Omega = m-1$. 
        \item $p_i(\hat{\theta})$ denpends on $k$-dimensional parameter $\theta$ so $\dim\omega_0 = k$
    \end{itemize}
    The large sample theory, tells us that the distribution of $-2\log\Lambda$ is $\chi^2_{m-k-1}$. 
\end{example}

\begin{definition}{\textbf{(Peason's Chi-Square Statistics)}}
    It is a commonly used to test for goodness of fit, where:
    \begin{equation*}
        X^2 = \sum^m_{i=1}\frac{\brackb{x_i - np_i(\hat{\theta})}^2}{np_i(\hat{\theta})}
    \end{equation*}
\end{definition}

\begin{proposition}
    Peason's statistics and likelihood tests are asymptoticall equivalent under $H_0$
\end{proposition}
\begin{proof}{\emph{(Sketch)}}
    Starting with the value:
    \begin{equation*}
        -2\log\Lambda = 2n\sum^m_{i=1}\hat{p}_i\log\bracka{\frac{\hat{p}_i}{p_i(\hat{\theta})}}
    \end{equation*}
    If $H_0$ is true and $n$ is large, then $\hat{p}_i\approx p_i(\hat{\theta})$. Consider the following Taylor series expansion of:
    \begin{equation*}
    \begin{aligned}
        f(x) &= x\log\bracka{\frac{x}{x_0}} \\
        &= (x-x_0) + \frac{1}{2}(x-x_0)^2\frac{1}{x_0} + \cdots 
    \end{aligned}
    \end{equation*}
    Thus, we have:
    \begin{equation*}
        -2\log\Lambda \approx 2n\sum^m_{i=1}[\hat{p}_i - p_i(\hat{\theta})] + n\sum^m_{i=1}\frac{[\hat{p}_i - p_i(\hat{\theta})]^2}{p_i(\hat{\theta})}
    \end{equation*}
    The first term is zero due to the fact that probabilities sum to $1$, while the second terms is equal to Peason's statistics. Note that Peason's statistics is easier to calculate than the likelihood ratio test.
\end{proof}

\begin{example}{\textbf{(Poisson Dispersion Test)}}
    Gives counts $x_1,\dots,x_n$, we consider:
    \begin{itemize}
        \item $H_0$: The counts are poisson with common parameter $\lambda$. Under $\omega_0$ the MLE of $\lambda$ is $\hat{\lambda} = \bar{X}$. 
        \item $H_1$: The counts have difference rates $\lambda_1,\dots,\lambda_n$. Under $\Omega$ we have $\tilde{\lambda_i} = x_i$ 
    \end{itemize}
    Please note that $\omega_0 \subset\Omega$ is the speical case that they are all equal, so the likelihood ratio is:
    \begin{equation*}
    \begin{aligned}
        &\Lambda = \cfrac{\prod^n_{i=1}\hat{\lambda}^{x_i}\cfrac{\exp(-\hat{\lambda})}{x_i!}}{\prod^n_{i=1}\tilde{\lambda}^{x_i}\cfrac{\exp(-\tilde{\lambda})}{x_i!}} = \prod^n_{i=1}\bracka{\frac{\bar{x}}{x_i}}^{x_i}\exp(x_i-\bar{x}) \\
        \iff&\begin{aligned}[t]
            -2\log\Lambda &= -2\sum^n_{i=1}\brackb{x_i\log\bracka{\frac{\bar{x}}{x_i}} + (x_i - \bar{x})} \\
            &=2\sum^n_{i=1}x_i\log\bracka{\frac{x_i}{\bar{x}}}
        \end{aligned}
    \end{aligned}
    \end{equation*}
    We have the following dimensions for the parameter spaces:
    \begin{itemize}
        \item $\Omega$, there are $n$ independent parameter $\lambda_1,\dots,\lambda_n$ so $\dim\Omega=n$.
        \item $\omega_1$, there is only one parameter so $\dim\omega = 1$
    \end{itemize}
    Thus, the test statistics distribution is $\chi^2_{n-1}$. We can interprete the test statistics as the ratio of $n$ times the estimated variance to estimated mean. 
\end{example}
\begin{remark}
    We can use Taylor series argument to approximate the test statistics for poisson dispersion test:
    \begin{equation*}
        -2\log\Lambda \approx \frac{1}{\bar{x}}\sum^n_{i=1}(x_i - \bar{x})^2
    \end{equation*}
\end{remark}

\subsection{Testing via Plotting}

\begin{definition}{\textbf{(Hanging Rootograms)}}
    Graphical display of the differences between observed and fitted values in historgram. There are multiple sections of rootograms:
    \begin{itemize}
        \item \textbf{Compare Observed Quatities:} We want to compare the observed frequencies with the frequencies fit by the normal distribution. Given the parameters are approximated as $\mu \approx \bar{x}$ and $\sigma \approx \hat{\sigma}$. If $j$-th interval has the left boundary $x_{j-1}$ and right boundary $x_j$. The probability falls in that interval is:
        \begin{equation*}
            \hat{p}_j = \Phi\bracka{\frac{x_j - \bar{x}}{\hat{\sigma}}} - \Phi\bracka{\frac{x_{j-1} - \bar{x}}{\hat{\sigma}}}
        \end{equation*}
        we can predict the count on $j$-th interval as $\hat{n}_j = n\hat{p}_j$, which can be compared to observed counts. Now, we can find the differences between the expected count and observed out. However, we neglet the variability in the estimated expected counts.
        \item \textbf{Variability:} If we neglect the variability in the estimated expected counts as we have:
        \begin{equation*}
            \operatorname{var}(n_j - \hat{n}_j) = \operatorname{var}(n_j) = np_j - np_j^2
        \end{equation*}
        if $p_j$ are small, we have $\operatorname{var}(n_j - \hat{n}_j) \approx np_j$. For a large values of $p_j$ have more varaible differences $n_j - \hat{n}_j$. And, so we expect larger fluctuation in the center than in the tails. 
        \item \textbf{Variance-Stabilizing Transfromation:} Suppose that a random variable $X$ has mean $\mu$ and variance $\sigma^2(\mu)$. If $Y = f(X)$, the method of propagation of error shows that:
        \begin{equation*}
            \operatorname{Var}(Y) \approx \sigma^2(\mu)[f'(\mu)]^2
        \end{equation*}
        If $f$ is chosen so that $\sigma^2(\mu)[f'(\mu)]^2$ is constant, the variance of $Y$ will not depends on $\mu$. Thus the transformation accomplishes variance-stabilizing transformation. 
        \item \textbf{Variability-Stabilizing:} Apply this to the case, and we have:
        \begin{equation*}
            \mathbb{E}[n_j] = np_j = \mu \qquad \operatorname{var}(n_j) \approx np_j = \sigma^2(\mu)
        \end{equation*}
        That is when $\sigma^2(\mu) = \mu$. The variance stabilizing transformation $\mu[f'(\mu)]^2$ should be $f(x) = \sqrt{x }$ does the job so:
        \begin{equation*}
            \mathbb{E}[\sqrt{n_j}] \approx \sqrt{np_j} \qquad \operatorname{var}(\sqrt{n_j}) \approx \frac{1}{4}
        \end{equation*}
        If the method is correct, and so we compare the differences as $\sqrt{n_j} - \sqrt{\hat{n}_j}$. 
        \item \textbf{Interpretation:} We use the deviation of more than $2$ and $3$ standard deviations is large. The run of positive deviations followed by the run of negative deviations and then the large positive deviation in the extreme right tail. This indicates some asymmetry in the distribution. 
        \item \textbf{Hanging Chi-Gram:} The plot of the components of Pearson's chi-square statistics:
        \begin{equation*}
            \frac{n_j - \hat{n}_j}{\sqrt{\hat{n}_j}}
            \implies \operatorname{var}\bracka{\frac{n_j-\hat{n}_j}{\sqrt{\hat{n}_j}}} \approx 1
        \end{equation*}
        Neglecting the variability in the expected counts, $\operatorname{var}(n_j-\hat{n}_j)\approx np_j = \hat{n}_j$, while the it is stabilizes the variance. This leads to the hanging $\chi^2$-gram.
    \end{itemize}
\end{definition}

\begin{definition}{\textbf{(Order Statistics)}}
    Consider the sample of size $n$ from a uniform distribution $[0, 1]$. The ordered sample values by $X_{(1)} < X_{(2)} < \cdots < X_{(n)}$. These values are called order statistics. 
\end{definition}

\begin{remark}{\textbf{(Understanding the Plots)}}
    We can show that:
    \begin{equation*}
        \mathbb{E}[X_{(j)}] = \frac{j}{n+1}
    \end{equation*}
    If the underlying distribution is uniform, the plot is shown in figure below, it is plotted for sample of size $100$ from a uniform distribution. Now, we consider the triangular distribution as we have:
    \begin{equation*}
        f(y) = \begin{cases}
            4y & 0 \le y \le \frac{1}{2} \\
            4-4y & \frac{1}{2} \le y \le 1 
        \end{cases}
    \end{equation*}
    THe ordered observation $Y_{1},\dots,Y_{100}$ are plotted against the points $1/(n+1),\dots,n/(n+1)$:
    \begin{figure}[H]
    \centering
    \begin{subfigure}{.5\textwidth}
        \centering
        \includegraphics[width=0.7\linewidth]{img/img1.png}
        \caption{Uniform-Uniform Probability Plot}
        \label{fig:1-sub1}
    \end{subfigure}%
    \begin{subfigure}{.5\textwidth}
        \centering
        \includegraphics[width=0.7\linewidth]{img/img2.png}
        \caption{Uniform-Triangular Probability Plot}
        \label{fig:1-sub2}
    \end{subfigure}
    \end{figure}
    We can see that there is a clear deviation from the linearity and allow us to describe qualitatively the deviation of the distribution of $Y$'s from the uniform distribution:
    \begin{itemize}
        \item The left tail of the plotted distribution are larger than the expected for a uniform distribution
        \item The right tail is smaller, which tells us that the distribution of $Y$ decreases more quickly than the tails of the uniform distribution. 
    \end{itemize}
\end{remark}

\begin{definition}{\textbf{(Probability Integral Transform)}}
    The technique can be extended to other continuous probability.
    If $X$ is a continuous random variable with a strictly increasing cumulative distribution function, and if $Y = F_X(X)$, then $Y$ has a uniform distribution on $[0, 1]$, as:
    \begin{equation*}
        P(Y \le y) = P(F_X(X) \le y) = P(X \le F^{-1}(y)) = F(F^{-1}(y)) = y
    \end{equation*} 
    This is the uniform of cdf. This transformation is known as probability integral transform.
\end{definition}

\begin{remark}{\textbf{(Probability Plot)}}
    Suppose that it is hypothesized that $X$ follows a certain distribution $F$. Given a sample $X_1,\dots,X_n$, we plot:
    \begin{equation*}
    \begin{aligned}
        F(X_{(k)}) \quad \text{ vs } \quad \frac{k}{n+1} \quad \implies \qquad X_{(k)} \quad\text{ vs }\quad F^{-1}\bracka{\frac{k}{n+1}}
    \end{aligned}
    \end{equation*}
    In some cases, $F$ is of the form $F(X) = G\bracka{\frac{x-\mu}{\sigma}}$, where $\mu$ and $\sigma$ are location and scale parameter. The normal distribution is of this form, we could plot:
    \begin{equation*}
        \frac{X_{(k)} - \mu}{\sigma} \quad \text{ vs }  \quad G^{-1}\bracka{\frac{k}{n+1}}
    \end{equation*}
    or if we plot $X_{(k)}$ vs $G^{-1}\bracka{\frac{k}{n+1}}$. The result would be approixmately a straight line if the model were correct:
    \begin{equation*}
        X_{(k)} \approx \sigma G^{-1}\bracka{\frac{k}{n+1}} + \mu
    \end{equation*}
\end{remark}

\begin{remark}{\textbf{(Slight Modification)}}
    Slight modification of this procedure are sometimes used. For example $\mathbb{E}[X_{(k)}]$ is used instead, as we have:
    \begin{equation*}
        \mathbb{E}[X_{(k)}] \approx F^{-1}\bracka{\frac{k}{n+1}} = \sigma G^{-1}\bracka{\frac{k}{n+1}} + \mu
    \end{equation*}
    The modification yields similar result to the original procedure. 
\end{remark}

\begin{remark}{\textbf{(Another Interpretation)}}
    Recall that $F^{-1}[k/(n+1)]$ is the $k/(n+1)$ quantile of the distribution $F$, the point such that the probability that a random variable with distribution function $F$ is less than it is $k/(n+1)$. We are plotting the ordered observations versus the quantile of the theoretical distribution. 
\end{remark}

\subsection{Testing for Normality}

\begin{definition}{\textbf{(Coefficient of Skewness)}}
    The skewness is usually characterized by the third central moments as:
    \begin{equation*}
        \int^\infty_{i\infty}(x-\mu)^2\varphi(x)\dby x
    \end{equation*}
    which is equal to $0$ given the normal distribution. Now, coefficient of skewness is:
    \begin{equation*}
        b_1 = \frac{1}{ns^3}\sum^n_{i=1}(X_i-\bar{X})^3
    \end{equation*}
\end{definition}

\begin{definition}{\textbf{(Coefficient of Kurtosis)}}
    Symmetric distribution can depart from normality by being heavy tailed or light-tailed. This is characterized by coefficient of Kurtosis as:
    \begin{equation*}
        b_2 = \frac{1}{ns^4}\sum^n_{i=1}(X_i-\bar{X})^4
    \end{equation*}
\end{definition}

\begin{remark}{\textbf{(Test for Normality)}}
    We can use both coefficient for skewness and kurtosis to access the normality of the data. Otherwise, we can use the hypothesis test, but is are difficult to evaluate in closed form but can be approximated by simulation.
\end{remark}


\section{Summarizing Data}

\subsection{Methods Based on CDF}

\begin{definition}{\textbf{(Empirical CDF)}}
    Suppose we have $x_1,\dots,x_n$ be a batch of numbers. The empirical cumulative distribution function is defined as:
    \begin{equation*}
        F_n(x) = \frac{1}{n}(\# x_i \le x)
    \end{equation*}
    Or, we have an ordered number of $x_{(1)}\le x_{(2)} \le \cdots \le x_{(n)}$. We have: if $x_{(k)} \le x < x_{(k+1)}$, then $F_n(x) = k/n$. 
\end{definition}

\begin{remark}{\textbf{(Comments on Empirical CDF)}}
    In the analysis, it is better to express $F_n$ in the following way, given random variables $X_1,\dots,X_n$:
    \begin{equation*}
        F_n(x) = \frac{1}{n} \sum^n_{i=1} I_{(-\infty, x]}(X_i) \qquad \text{ where } \qquad I_{(-\infty, x]}(X_i) = \begin{cases}
            1 & \text{ if } X_i \le x \\
            0 & \text{ otherwise }
        \end{cases}
    \end{equation*}
    The random variable $I_{(-\infty, x]}(X_i)$ are independent Bernoulli random variables, where we have:
    \begin{equation*}
        I_{(-\infty, x]}(X_i) = \begin{cases}
            1 & \text{ with probability } F(x) \\
            0 & \text{ with probability } 1-F(x) \\
        \end{cases}
    \end{equation*}
    Thus, $nF_n(x)$ is a binomial random variable ($n$ trials with probability of $F(x)$ of success), as we have:
    \begin{equation*}
        \mathbb{E}[F_n(x)] = F(x) \qquad \operatorname{var}(F_n(x)) = \frac{1}{n}F(x)[1-F(x)]
    \end{equation*}
    An estimate of $F_n(x)$ is unbiased and has a maximum variacne at the value of $x$ such that $F(x) = 0.5$, which is at median. 
\end{remark}

\begin{remark}{\textbf{(Behavior of $\boldsymbol F_n$)}}
    If we consider the stochastic behavior of $F(x)$, then we can show that:
    \begin{equation*}
        \max_{-\infty<x<\infty} \abs{F_n(x) - F(x)}
    \end{equation*}
    doesns't depend on $F$ if $F$ is continuous. This allow us to construct a simultaneous confidence band about $F_n$, which can be used to test goodness-of-fit. Please note that this isn't the same compared to the confidence interval of binomial distribution. 
\end{remark}

\begin{definition}{\textbf{(Survival Function)}}
    It is equivalent to CDF and is defined as:
    \begin{equation*}
        S(t) = \mathbb{P}(T > t) = 1-F(t)
    \end{equation*}
    where $T$ is a random variable with CDF of $F$. We use it where the data consists of times until failure or death and so non-negative. $S(t)$ denotes the lifetime will be longer than $t$, and so we can have empirical version to be $S_n(t) = 1-F_n(t)$. 
\end{definition}

\begin{definition}{\textbf{(Hazard Function)}}
    It is interpreted as the instantaneous death rate for individual who have survived up to a given time. If an individual is alive at time $t$, the probability that the individual will die at time interval $(t, t + \delta)$ is (assuming density function $f$ is continuous at $t$):
    \begin{equation*}
    \begin{aligned}
        P(t \le T \le t + \delta | T\ge t) &= \frac{P(t\le T \le t + \delta)}{P(T \ge t)} \\
        &= \frac{F(t + \delta) - F(t)}{1 - F(t)} \approx \frac{\delta f(t)}{1-F(t)}
    \end{aligned}
    \end{equation*}
    The hazard function is defined as:
    \begin{equation*}
        h(t) = \frac{f(t)}{1-F(t)}
    \end{equation*}
    If $T$ is the lifetime of a manufactured component, it may be natural to think of $h(t)$ as the instantaneous or age-specific failure rate. 
\end{definition}

\begin{remark}{\textbf{(Interpretation of Hazard Function)}}
    It can be expressed as:
    \begin{equation*}
        h(t) = -\frac{d}{dt}\log[1-F(t)] = -\frac{d}{dt} \log S(t)
    \end{equation*}
    Which is the negative of the log of survival funcion. With the method of propagation of error:
    \begin{equation*}
        \operatorname{var}\Big( 1 - F_n(t) \Big) \approx \frac{\operatorname{var}[1-F_n(t)]}{(1-F(t))^2} = \frac{1}{n}\bracka{\frac{F(t)}{1-F(t)}}
    \end{equation*}
    For large value of $t$, the empirical log survial function is unrealiable, because $1-F(t)$ is very small, and so in practice, last few data are disregarded.
\end{remark}

\begin{remark}{\textbf{(Empirical Survial Function)}}    
    Suppose that there are no ties and the ordered failure times are: $T_{(1)} < T_{(2)} < \cdots < T_{(n)}$. If $t = T_{(i)}$, $F_n(t) = i/n$ and $S_{n}(t) = 1-i/n$. But since $\log S_n(t)$ is undefined for $t\ge T_{(n)}$, it is ofen defined as:
    \begin{equation*}
        S_n(t) = 1 - \frac{i}{n+1}
    \end{equation*}
    for $T_{(i)} \le t < T_{(i+1)}$
\end{remark}

\begin{definition}{\textbf{(Quantile-Qunatile Plot)}}
    If $X$ is a continuous random variable with a strictly increasing distribution function $F$, the $p$-th quantile of the to be value of $x$ such that: $F(x) = p$ or $x_p = F^{-1}(p)$. In Q-Q plot, the quantile of one distribution is plotted against another. 
\end{definition}

\begin{remark}{\textbf{(Usage of Q-Q)}}
    Suppose we have $2$ distributions:
    \begin{itemize}
        \item $F$ is a model for observations of a control group. 
        \item $G$ is a model for observations of a group that has received some treatment. 
    \end{itemize}
    Let's consider how difference update changes the plot:
    \begin{itemize}
        \item Suppose that there is an effect changned by $h$ uniformly i.e $y_p = x_p + h$, where $y_p$ is the group that received the treatment and vice versa. This gives us the relationship to be: $G(y) = F(y - h)$. 
        \item Similarly, we have the effect with multiplicative differences i.e given $c \in \mathbb{R}$ where we have $y_p = cx_p$ with the relationship to be $G(y) = F(y/h)$
    \end{itemize}
    Given the number of samples, we have to use the empirical CDF to create thE Q-Q plot. Now, the results of the changes is shown in the following figure:
    \begin{figure}[H]
    \centering
    \begin{subfigure}{.5\textwidth}
        \centering
        \includegraphics[width=0.7\linewidth]{img/img3.png}
        \caption{Additive Treatment Effect}
        \label{fig:1-sub1}
    \end{subfigure}%
    \begin{subfigure}{.5\textwidth}
        \centering
        \includegraphics[width=0.7\linewidth]{img/img4.png}
        \caption{Multiplicative Treatment Effect}
        \label{fig:1-sub2}
    \end{subfigure}
    \end{figure}
\end{remark}

\begin{definition}{\textbf{(Kernel Probability Density Estimate)}}
    Let $w(x)$ be a non-negative, symmetric weight function, centered at zero and integrating to $1$. It can be standard normal density, with the following rescaled version:
    \begin{equation*}
        w_h(x) = \frac{1}{h}w\bracka{\frac{x}{h}}
    \end{equation*}
    is a rescaled version of $w$, as it approaches zero, $w_h$ becomes more concentrated and peaked around zero. On the other hand, as $h$ approaches infinity, $w_h$ becomes flat. If $X_1,\dots,X_n$ is a sample from a probability density function $p$, its esitmate is:
    \begin{equation*}
        f_h(x) = \frac{1}{n}\sum^n_{i=1}w_h(x - X_i)
    \end{equation*}
    The parameter $h$ represents bandwidth of estimating function as it controls the smoothness.
\end{definition}

\subsection{Meansure of Location}

\begin{definition}{\textbf{(Arithmetic Mean)}}
    The commonly used measure of location is the arithmetic mean, which is:
    \begin{equation*}
        \bar{x} = \frac{1}{n}\sum^n_{i=1}
    \end{equation*}
\end{definition}

\begin{remark}{\textbf{(Problem with Arithmeic Mean)}}
    By changing a single number, the arithmetic mean of a batch of numbers can be made arbitary large or smaller. Thus, when used blindly, without careful attention, the mean can produce a misleading results. Or, we need to have the measure of location that are robut or insensitive to outlier. 
\end{remark}

\begin{remark}{\textbf{(Why Sample Mean is Bad)}}
    The sample mean minimizers the log-likelihood of:
    \begin{equation*}
        \sum^n_{i=1}\bracka{\frac{(X_i - \mu)^2}{\sigma}}
    \end{equation*}
    This is the simpliest case of least square estimate. The outlier have a great effect on this estimate, as the deviation of $\mu$ from $X_i$ is measured by square of their difference.
\end{remark}

\begin{definition}{\textbf{(Median)}}
    It is a middle value of the ordered observation; if the sample size is even, the median is the average of the $2$ middle values. 
\end{definition}

\begin{proposition}{\textbf{(Confidence Interval)}}
    We can show that, given the population median $\eta$ and the interval between the order statistics $(X_{(k)}, X_{(n-k+1)})$
    \begin{equation*}
        P(X_{(k)} \le \eta \le X_{(n-k+1)}) = 1 - \frac{1}{2^{n-1}}\sum^{k-1}_{j=0}
    \end{equation*}
\end{proposition}
\begin{proof}
    The coverage probability of this interval is:
    \begin{equation*}
    \begin{aligned}
        P(X_{(k)} \le \eta \le X_{(n-k+1)}) &= 1 - P(\eta < X_{(k)} \text{ or } \eta > X_{n-k+1}) \\
        &= 1 - P(\eta < X_{(k)}) - P(\eta > X_{(n-k+1)})
    \end{aligned}
    \end{equation*}
    Since the event are mutually exclusive. To evaluate both terms, we note that:
    \begin{equation*}
    \begin{aligned}
        &P(\eta > X_{(n-k+1)}) = \sum^{k-1}_{j=0} \mathbb{P}(j \text{ observations} > \eta) \\
        &P(\eta < X_{(k)}) = \sum^{k-1}_{j=0} \mathbb{P}(j \text{ observations } < \eta)
    \end{aligned}
    \end{equation*}
    The median satisfies $P(X_i > \eta ) = P(X_i < \eta) = 1/2$, since $n$ observations $X_1,\dots,X_n$ are independent and identically distributed, the distribution of the number of observation greater than median is binomial with $n$ trials and probability $1/2$:
    \begin{equation*}
        P(j \text{ observations } > \eta) = \frac{1}{2}\begin{pmatrix}
            n \\ j
        \end{pmatrix}
    \end{equation*}
    and, so we have:
    \begin{equation*}
        P(\eta > X_{(n-k+1)}) = \frac{1}{2^n}\sum^{k-1}_{j=0}\begin{pmatrix}
            n \\ j
        \end{pmatrix}
    \end{equation*}
    This is the same for $P(\eta < X_{(k)})$ due to symmetry. Plugging it back to finish the proof
\end{proof}

\begin{remark}
    Median can be seen as the minimizer of the following loss:
    \begin{equation*}
        \sum^n_{i=1}\abs{\frac{X_i - \mu}{\sigma}}
    \end{equation*}
    Here, large deviation are not weighted as heavily, making median robust. The proof follows from the fact that the dervative of absolute is $\operatorname{sgn}(\cdot)$, and so the loss is zero when the positive $x - \mu$ (of the normalized data ) is equal to the negative item $x - \mu$, which is where the median situates. 
\end{remark}

\begin{definition}{\textbf{(Trimmed Mean)}}
    The $100\alpha\%$ trimmed mean consider the valuse that is between the lower $100\alpha\%$ and the higher $100\alpha\%$, as we can write it as:
    \begin{equation*}
        \bar{x}_\alpha = \frac{x_{[n\alpha] + 1} + \cdots + x_{(n - [n\alpha])}}{n - 2[n\alpha]}
    \end{equation*}
    where $[n\alpha]$ denotes the greatest integer less than or equal to $n\alpha$.
\end{definition}

\begin{definition}{\textbf{(M-Estimates)}}
    Consider the class of esitmates called $M$-estimates, where it is a minimizer:
    \begin{equation*}
        \sum^n_{i=1}\Psi\bracka{\frac{X_i - \nu}{\sigma}}
    \end{equation*}
    where $\Psi$ is the weight function that is a compromise between weight function for mean and median. 
\end{definition}

\begin{remark}{\textbf{(Measure of Dispersion)}}
    The most commonly used measure is sample standard deviation, where it is given as:
    \begin{equation*}
        S^2 = \frac{1}{n-1}\sum^n_{i=1}(X_i - \bar{X})^2
    \end{equation*}
    Using $n-1$ as divisor gives unbiased estimate. But like a sample mean standard deviation is sensitive to outlying observation. Two simple robust measures alternative are:
    \begin{itemize}
        \item Interquartile range (IQR): Differences between $2$ sample quantiles.
        \item Median absolute deviation from the median (MAD): If data are $x_1,\dots,x_n$ with median $\tilde{x}$, then MAD is the median of number $\abs{x_1,\dots,x_n}$. 
    \end{itemize}
\end{remark}



\section{Belief Propagation: Interpretation}

\subsection{Introduction}

\begin{definition}{\textbf{(Loopy Propagation)}}
    The joint distribution for \emph{any} graph is given by:
    \begin{equation*}
        P(\mathcal{X}) = \frac{1}{Z}\prod_{\text{nodes } i} f_i(\boldsymbol x_i) \prod_{\text{edges } (ij)} f_{ij}(\boldsymbol x_i, \boldsymbol x_j)
    \end{equation*}
    Message computed recusively with few guarantee of convergence as we have the following message:
    \begin{equation*}
        M_{j\rightarrow i} = \sum_{\boldsymbol x_j} f_{ij}(\boldsymbol x_i, \boldsymbol x_j)f_j(\boldsymbol x_j)\prod_{l \in \operatorname{ne}(j)\backslash\brackc{i}} M_{l\rightarrow j}(\boldsymbol x_j)
    \end{equation*}
    The marginal distribution are approximation in general:
    \begin{equation*}
    \begin{aligned}
        &P(\boldsymbol x_i) \approx b_i(\boldsymbol x_i) \propto f_i(\boldsymbol x_i)\prod_{k\in\operatorname{ne}(i)}M_{k\rightarrow i}(\boldsymbol x_i) \\
        &P(\boldsymbol x_i, \boldsymbol x_j) \approx b_{ij}(\boldsymbol x_i, \boldsymbol x_j) \propto f_{ij}(\boldsymbol x_i, \boldsymbol x_j)f_i(\boldsymbol x_i)f_j(\boldsymbol x_j)\prod_{k\in\operatorname{ne}(i)\backslash \brackc{j}} M_{k\rightarrow i}(\boldsymbol x_i)\prod_{l\in\operatorname{ne}(j)\backslash \brackc{i}}M_{l\rightarrow j}(\boldsymbol x_j)
    \end{aligned}
    \end{equation*}
\end{definition}

\begin{remark}{\textbf{(Dealing with Loops)}}
    There are various way to deal with loop as we have:
    \begin{itemize}
        \item The belief propagation posterior marginal are approximate on all non-tree because over-counted, but converged approximate are frequently found to be good. 
        \item Converge can be seen in: Tree, Graph with single step, Distribution with weak iteraction, Graph with long (and weak) loops, and Gaussian network (variance may also converged).
        \item Damping, as it is a common approach to encorate of EP:
        \begin{equation*}
            M^\text{new}_{i\rightarrow j}(\boldsymbol x_j) = (1-\alpha)M^\text{old}_{i\rightarrow j} + \alpha\sum_{\boldsymbol x_i}f_{ij}(\boldsymbol x_i, \boldsymbol x_j)f_i(\boldsymbol x_i) \prod_{k\in\operatorname{ne}(i)\backslash\brackc{j}}M_{k\rightarrow i}(\boldsymbol x_i)
        \end{equation*}
        \item Variable can be groupped into cliques to improve accuracy: region graph approximate, cluster variable method, and junction graph.
    \end{itemize} 
\end{remark}

\subsection{Message Based EP}

\begin{proposition}{\textbf{(Loopy BP as Message-Based EP)}}
    One can consider the connection between message-based EP and loopy BP, as they are equivalent.
\end{proposition}
\begin{proof}
    Consider the appoximate pairwise factor $\tilde{f}_{ij}$ as product of messages:
    \begin{equation*}
        f_{ij}(\boldsymbol x_i, \boldsymbol x_j) \approx \tilde{f}_{ij}(\boldsymbol x_i, \boldsymbol x_j) = M_{i\rightarrow j}(\boldsymbol x_j)M_{j\rightarrow i}(\boldsymbol x_i)
    \end{equation*}
    Consider the approximation of the factorized distribution:
    \begin{equation*}
    \begin{aligned}
        P(\mathcal{X}) &\approx \frac{1}{Z}\prod_{\text{nodes}(i)}f_i(\boldsymbol x_i) \prod_{\text{edges}(ij)}\tilde{f}_{ij}(\boldsymbol x_i, \boldsymbol x_j) \\
        &= \frac{1}{Z}\prod_{\text{nodes}(i)}\bracka{f_i(\boldsymbol x_i)\prod_{j\in\mathcal{N}(i)}M_{j\rightarrow i}(\boldsymbol x_i)} = \prod_{\text{nodes}(i)}b_i(\boldsymbol x_i)
    \end{aligned}
    \end{equation*}
    with multiple factors for $\boldsymbol x_i$, which we consider the update on EP to be:
    \begin{itemize}
        \item \emph{Deletion}: Consider the following $P(\mathcal{X})$ as we have:
        \begin{equation*}
        \begin{aligned}
            P_{\neg ij}&(X_i, X_j) = \sum_{c \ne i, j} \frac{P(\mathcal{X})}{\tilde{f}(X_i, X_j)} = \sum_{c \ne i, j} \frac{P(\mathcal{X})}{ M_{i\rightarrow j}(\boldsymbol x_j) M_{j\rightarrow i}(\boldsymbol x_i)} \\ 
            &= \frac{1}{\tilde{f}(X_i, X_j)}\sum_{c \ne i, j}f_i(\boldsymbol x_i)f_j(\boldsymbol x_j)\prod_{k\in\mathcal{N}(i)}M_{k\rightarrow i}(\boldsymbol x_i)\prod_{l\in\mathcal{N}(j)}M_{l\rightarrow j}(\boldsymbol x_j) \bracka{\prod_{s\ne i, j} f_s(\boldsymbol x_s) \prod_{t\in\mathcal{N}(s)}  M_{t\rightarrow s} (\boldsymbol x_s) } \\
            &= f_i(\boldsymbol x_i)f_j(\boldsymbol x_j)\prod_{k\in\mathcal{N}(i) \backslash j}M_{k\rightarrow i}(\boldsymbol x_i)\prod_{l\in\mathcal{N}(j)\backslash i}M_{l\rightarrow j}(\boldsymbol x_j)\sum_{c \ne i, j}\bracka{\prod_{s\ne i, j} f_s(\boldsymbol x_s) \prod_{t\in\mathcal{N}(s)}  M_{t\rightarrow s} (\boldsymbol x_s) } \\
            &= f_i(\boldsymbol x_i)f_j(\boldsymbol x_j)\prod_{k\in\mathcal{N}(i) \backslash j}M_{k\rightarrow i}(\boldsymbol x_i)\prod_{l\in\mathcal{N}(j)\backslash i}M_{l\rightarrow j}(\boldsymbol x_j) \\
        \end{aligned}
        \end{equation*}
        \item \emph{Projection}: We consider minimizing the KL-divergence as we have:
        \begin{equation*}
            \brackc{M^\text{new}_{i\rightarrow j}, M^\text{new}_{j\rightarrow i}} = \argmin{M_{i\rightarrow j}, M_{j\rightarrow i}} \operatorname{KL}\brackb{ f_{ij}(\boldsymbol x_i, \boldsymbol x_j) q_{\neg ij}(\boldsymbol x_i, \boldsymbol x_j) \Big\| M_{j\rightarrow i}(\boldsymbol x_i) M_{i\rightarrow j}(\boldsymbol x_j)q_{\neg ij}(\boldsymbol x_i, \boldsymbol x_j) }
        \end{equation*}
        To solve this KL-divergence, this is obvious, as $q_{\neg ij}(\cdot)$ can be factorized and so the minimizer is the marginal between $f_{ij}(\cdot)q_{\neg ij}(\cdot)$, which means that:
        \begin{equation*}
        \begin{aligned}
            M_{i\rightarrow j}^\text{new}(\boldsymbol x_j)q_{\neg ij}(\boldsymbol x_i, \boldsymbol x_j)& = \sum_{\boldsymbol x_j}\bracka{f_{ij}(\boldsymbol x_i, \boldsymbol x_j) f_j(\boldsymbol x_j)\prod_{l\in\mathcal{N}(j)\backslash i}M_{l\rightarrow j}(\boldsymbol x_j) } \underbrace{f_i(\boldsymbol x_i)\prod_{k\in\mathcal{N}(i) \backslash j}M_{k\rightarrow i}(\boldsymbol x_i)}_{q_{\neg ij}(\boldsymbol x_i)} \\
            \implies& M_{i\rightarrow j}^\text{new}(\boldsymbol x_j) = \sum_{\boldsymbol x_j}\bracka{f_{ij}(\boldsymbol x_i, \boldsymbol x_j) f_j(\boldsymbol x_j)\prod_{l\in\mathcal{N}(j)\backslash i}M_{l\rightarrow j}(\boldsymbol x_j) }
        \end{aligned}
        \end{equation*}
        This is the Loopy BP update, and so both of the are equivalent. 
    \end{itemize}
\end{proof}

\begin{remark}{\textbf{(Comments on the Loopy BP and EP)}}
    There are some observation that we can make in the equivalent between loopy BP and EP algorithm:
    \begin{itemize}
        \item Unlike EP, this message based EP doesn't need $2$ separate approximate as we have in the normal EP.
        \item This message based EP is loopy graph can be seen as a more constraint on approximate site and not just exponential family factor but the product of exponential family message. 
        \item On a tree, message forward EP finds the same marginal as standard EP as the messages are calculated the same way. Similarly, the pairwise marginal can be found after converge by compute $\tilde{P}(z_{i - 1}, z_i)$
        \item Factorization still remain valid even when original site lies in the appoximation exponential family already, so the loopy BP can be seen as form of EP. 
        \item This doesn't help us with understanding the convergence property of EP.
    \end{itemize}
\end{remark}

\subsection{Reparameterized on Tree}

\begin{remark}{\textbf{(Tree-Based Representation)}}
    We consider the joint factorization, which can be represented:
    \begin{equation*}
    \begin{aligned}
        P(\mathcal{X}) &= \frac{1}{Z}\prod_{\operatorname{nodes}(i)} f_i(\boldsymbol x_i)\prod_{\operatorname{edges}(ij)} f_{ij}(\boldsymbol x_i, \boldsymbol x_j) \qquad \text{(Undirected Tree)} \\
        &= P(\boldsymbol x_i)\prod_{i\ne r}P(\boldsymbol x_i | \boldsymbol x_{\text{pa}(i)}) \qquad \text{(Directed Rooted Tree)} \\
        &= \prod_{\operatorname{nodes}(i)}P(\boldsymbol x_i)\prod_{\operatorname{edges}(ij)}\frac{P(\boldsymbol x_i, \boldsymbol x_j)}{P(\boldsymbol x_i)P(\boldsymbol x_j)}  \qquad \text{(Pairwise Marginal)} \\
    \end{aligned}
    \end{equation*}
    The last on requires that $\sum_{\boldsymbol x_j}P(\boldsymbol x_i, \boldsymbol x_j) = P(\boldsymbol x_i)$. 
    \begin{itemize}
        \item The unidrected tree isn't unique as if we multiply the factor $f_{ij}(\boldsymbol x_i, \boldsymbol x_j)$ by $g(\boldsymbol x_i)$ and dividing $f_i(\boldsymbol x_i)$ by the same $g(\boldsymbol x_i)$ doesn't change the distribution. 
        \item BP can be seen as iteractive replacement of $f_i(\boldsymbol x_i)$ by local marginal of $p_{ij}(\boldsymbol x_i, \boldsymbol x_j)$ along with corresponding representation of $f_{ij}(\boldsymbol x_i, \boldsymbol x_j)$ (recall the Hugin propagation)
        \item Converged BP on a tree finds $P(\boldsymbol x_i)$ and $P(\boldsymbol x_i, \boldsymbol x_j)$ allowing up to transform the undirected tree to pairwise marginal. 
    \end{itemize}
\end{remark}

\begin{remark}{\textbf{(Reparameterization in Tree)}}
    To consider the tree based reparameterization, we want to transform the representation from undirected tree to pairwise marginal as:
    \begin{equation*}
        \prod_{\operatorname{nodes}(i)} f_i(\boldsymbol x_i)\prod_{\operatorname{edges}(ij)} f_{ij}(\boldsymbol x_i, \boldsymbol x_j) \implies \prod_{\operatorname{nodes}(i)}P(\boldsymbol x_i)\prod_{\operatorname{edges}(ij)}\frac{P(\boldsymbol x_i, \boldsymbol x_j)}{P(\boldsymbol x_i)P(\boldsymbol x_j)}
    \end{equation*}
    We will define the $f^0_{ij} = f_{ij}$, while the singleton factor to be $f^0_1 = p^0_1 = 1$, we consider the following update: The update is based on the fact that we will act on the factors \emph{as if} it is actually representing the probabilities: We will consider such a procedure on a node that has $2$ incoming messages. 
    \begin{itemize}
        \item Starting with joint, where if we multiply it by adjacent factors we get $P(\boldsymbol x_i, \boldsymbol x_j)$ i.e
        \begin{equation*}
            p^{(n)}(\boldsymbol x_i, \boldsymbol x_j) = \frac{1}{Z_{ij}^{(n)}} f^{(n-1)}_i(\boldsymbol x_i)f^{(n-1)}_{ij}(\boldsymbol x_i, \boldsymbol x_j) f^{(n-1)}_j(\boldsymbol x_j)
        \end{equation*}
        \item Finding the marginal, as we have:
        \begin{equation*}
            f^{(n)}_i(\boldsymbol x_i) = p^{(n)}(\boldsymbol x_i) = \sum_{\boldsymbol x_j}p^{(n)}(\boldsymbol x_i, \boldsymbol x_j) = f^{(n-1)}_i(\boldsymbol x_i)\underbrace{\sum_{\boldsymbol x_j} f^{(n-1)}_{ij}(\boldsymbol x_i, \boldsymbol x_j) f^{(n-1)}_j(\boldsymbol x_j)}_{M_{j\rightarrow i}}
        \end{equation*}
        \item To keep the normalization correctly, we divide the message so that the update on one passing giving us normalized term:
        \begin{equation*}
            f^{(n)}_{ij} = \frac{f^{(n-1)}_{ij}(\boldsymbol x_i, \boldsymbol x_j)}{M_{j\rightarrow i}(\boldsymbol x_j)}
        \end{equation*}
        \item We now consider the next step with the next incoming message from node $k$ to node $i$:
        \begin{equation*}
        \begin{aligned}
            p^{(n)}(\boldsymbol x_i, \boldsymbol x_k) &= \frac{1}{Z_{ik}^{(n)}} f^{(n)}_i(\boldsymbol x_i)f^{(n-1)}_{ik}(\boldsymbol x_i, \boldsymbol x_j) f^{(n-1)}_k(\boldsymbol x_j) \\
            &= \frac{1}{Z_{ik}^{(n)}} f^{(n-1)}_i(\boldsymbol x_i)M_{j\rightarrow i}(\boldsymbol x_i) f^{(n-1)}_{ik}(\boldsymbol x_i, \boldsymbol x_k) f^{(n-1)}_k(\boldsymbol x_k) \\
        \end{aligned}
        \end{equation*}
        \item Finding the singleton factor by marginalization
        \begin{equation*}
            f^{(n)}_i(\boldsymbol x_i) = f^{(n-1)}_i(\boldsymbol x_i)M_{j\rightarrow i}(\boldsymbol x_i) \underbrace{\sum_{\boldsymbol x_k}f^{(n-1)}_{ik}(\boldsymbol x_i, \boldsymbol x_k) f^{(n-1)}_k(\boldsymbol x_k)}_{M_{k\rightarrow i}}
        \end{equation*}
        \item And so, the normalization correction on the joint factor is:
        \begin{equation*}
            f^{(n)}_{ik} = \frac{f^{(n-1)}_{ik}(\boldsymbol x_i, \boldsymbol x_k)}{ M_{k\rightarrow i}(\boldsymbol x_i) }
        \end{equation*}
    \end{itemize}
    We perform this update throughout the tree, which we do it in forward (e.g $i\rightarrow j$) and backward (e.g $j\rightarrow i$) manner, which gives us:
    \begin{equation*}
    \begin{aligned}
        &f^{(\infty)}_i(\boldsymbol x_i) = \prod_{j\operatorname{ne}(i)}M_{j\rightarrow i}(\boldsymbol x_i) = P(\boldsymbol x_i) \\
        &\begin{aligned}[t]
            f^{(\infty)}_{ij}(\boldsymbol x_i, \boldsymbol x_j) &= \frac{f_{ij}(\boldsymbol x_i, \boldsymbol x_j)}{M_{j\rightarrow i}(\boldsymbol x_i)M_{i\rightarrow j}(\boldsymbol x_j)} \\
            &= \frac{\prod_{k \in \operatorname{ne}(i)\backslash j}M_{k\rightarrow i}f_{ij}(\boldsymbol x_i, \boldsymbol x_j)\prod_{l \in \operatorname{ne}(j)\backslash i}M_{l\rightarrow i}(\boldsymbol x_j)}{\prod_{k \in \operatorname{ne}(i)\backslash j}M_{k\rightarrow i} M_{j\rightarrow i}(\boldsymbol x_i)M_{i\rightarrow j}(\boldsymbol x_j) \prod_{l \in \operatorname{ne}(j)\backslash i}M_{l\rightarrow i}(\boldsymbol x_j) } \\
            &= \frac{\prod_{k \in \operatorname{ne}(i)\backslash j}M_{k\rightarrow i}f_{ij}(\boldsymbol x_i, \boldsymbol x_j)\prod_{l \in \operatorname{ne}(j)\backslash i}M_{l\rightarrow i}(\boldsymbol x_j)}{\prod_{k \in \operatorname{ne}(i)}M_{k\rightarrow i} \prod_{l \in \operatorname{ne}(j)}M_{l\rightarrow i}(\boldsymbol x_j) } \\
            &= \frac{P(\boldsymbol x_i, \boldsymbol x_j)}{P(\boldsymbol x_i)P(\boldsymbol x_j)}
        \end{aligned}
    \end{aligned}
    \end{equation*}
    the equation follows from the result from belief propagation.This kind of reparameterization allows us to avoid double counting, which is essentially a book-keeping method, espescially the normalizing part (there will be a case where the factor cancel with unnecessary message from singleton factor, as intended). 
\end{remark}

\begin{remark}{\textbf{(Comments on the BP on non-tree)}}
    If this converges in a non-tree setting, then we have locally consistent belief i.e:
    \begin{equation*}
        p(\mathcal{X}) \propto \prod_i b(\boldsymbol x_i) \prod_{ij}\frac{b(\boldsymbol x_i, \boldsymbol x_j)}{b(\boldsymbol x_i)b(\boldsymbol x_j)} \quad \text{ such that } \quad \sum_{\boldsymbol x_j}b(\boldsymbol x_i, \boldsymbol x_j) = b(\boldsymbol x_i)
    \end{equation*}
    But it doesn't need to be globally consistent:
    \begin{equation*}
        \sum_{\mathcal{X}_{\neg i}}\bracka{\prod_i b(\boldsymbol x_i) \prod_{ij}\frac{b(\boldsymbol x_i, \boldsymbol x_j)}{b(\boldsymbol x_i)b(\boldsymbol x_j)}} \ne b(\boldsymbol x_i)
    \end{equation*}
    This kind of marginal is called \emph{pseudo-marginal}. 
\end{remark}

\begin{remark}{\textbf{(Message Schedule Scheme)}}
    We consider update the belief on each \emph{subtree} of the graph and passing message on each subtree, looping through all the subtree until converge:
    \begin{figure}[H]
        \centering
        \includegraphics[width=10cm]{img/img16.png}
    \end{figure}  
    And, we now that the following updates steps:
    \begin{equation*}
    \begin{aligned}
        P(\mathcal{X}) 
        &= \frac{1}{Z}\prod_{\operatorname{nodes}(i)} f^{(0)}_i(\boldsymbol x_i)\prod_{\operatorname{edges}(ij)} f^{(0)}_{ij}(\boldsymbol x_i, \boldsymbol x_j) \\
        &= \frac{1}{Z}\prod_{\operatorname{nodes}(i) \in T_1} f^{(0)}_i(\boldsymbol x_i)\prod_{\operatorname{edges}(ij) \in T_1} f^{(0)}_{ij}(\boldsymbol x_i, \boldsymbol x_j)\bracka{\prod_{\operatorname{edges}(ij) \not\in T_1} f^{(0)}_{ij}(\boldsymbol x_i, \boldsymbol x_j)} \\
        &\text{(Update)} \\
        &= \frac{1}{Z}\prod_{\operatorname{nodes}(i) \in T_1} f^{(1)}_i(\boldsymbol x_i)\prod_{\operatorname{edges}(ij) \in T_1} f^{(1)}_{ij}(\boldsymbol x_i, \boldsymbol x_j)\bracka{\prod_{\operatorname{edges}(ij) \not\in T_1} f^{(1)}_{ij}(\boldsymbol x_i, \boldsymbol x_j)} \\
        &\text{(Next Tree)} \\
        &= \frac{1}{Z}\prod_{\operatorname{nodes}(i) \in T_2} f^{(1)}_i(\boldsymbol x_i)\prod_{\operatorname{edges}(ij) \in T_2} f^{(1)}_{ij}(\boldsymbol x_i, \boldsymbol x_j)\bracka{\prod_{\operatorname{edges}(ij) \not\in T_2} f^{(1)}_{ij}(\boldsymbol x_i, \boldsymbol x_j)} \\
        &\cdots
    \end{aligned}
    \end{equation*}
    where we have, when we got a new tree, 
    \begin{equation*}
        f^{(1)}_i(\boldsymbol x_i) = P^{T_1}(\boldsymbol x_i)\qquad f^{(1)}_{ij}(\boldsymbol x_i, \boldsymbol x_j) = \frac{P^{T_1}(\boldsymbol x_i, \boldsymbol x_j)}{P^{T_1}(\boldsymbol x_i)P^{T_2}(\boldsymbol x_j)} 
    \end{equation*}
    If the process converges, suppose it converge to:
    \begin{equation*}
        P(\mathcal{X}) = \frac{1}{Z}\prod_{\operatorname{nodes}(i)} f_i^{(\infty)}(\boldsymbol x_i)\prod_{\operatorname{edges}(ij)} f_{ij}^{(\infty)}(\boldsymbol x_i, \boldsymbol x_j)
    \end{equation*}
    where for any tree $T$ in the graph, we have:
    \begin{equation*}
        f_i^{(\infty)} = P^T(\boldsymbol x_i)\qquad f_{ij}^{(\infty)} = \frac{P^T(\boldsymbol x_i, \boldsymbol x_j)}{P^T(\boldsymbol x_i)P^T(\boldsymbol x_j)}
    \end{equation*}
    This means that the local marginal of all subtree are consistent with each other, and the pseudo-marginal is valid belief of any of the subtree, as this is stronger constriant. 
\end{remark}

\subsection{Bathe Free Energy}

\begin{remark}{\textbf{(Introduction to Bathe Free Energy)}}
    In reparameterization view, BP solves for marginal belief $b_{ij}(\boldsymbol x_i, \boldsymbol x_j)$ and $b_i(\boldsymbol x_i) = \sum_{\boldsymbol x_j}b_{ij}(\boldsymbol x_i, \boldsymbol x_j)$  such that:
    \begin{equation*}
        P(\mathcal{X}) \propto \prod_if_i(\boldsymbol x_i)\prod_{ij}f_{ij}(\boldsymbol x_i, \boldsymbol x_j) \propto \prod_ib_i(\boldsymbol x_i)\prod_{ij}\frac{b_{ij}(\boldsymbol x_i, \boldsymbol x_j)}{b_i(\boldsymbol x_i)b_j(\boldsymbol x_j)}
    \end{equation*}
    Loopy BP is a set of fixed point equation for finding stationary of an objective function called Bathe free energy, which is defined in terms of locally consistent belief (pseudo-marginal) $b_i\ge0$ and $b_{ij}\ge0$ such that:
    \begin{equation*}
        \sum_{\boldsymbol x_i} b_i(\boldsymbol x_i) = 1 \qquad \sum_{\boldsymbol x_j}b_{ij}(\boldsymbol x_i, \boldsymbol x_j) = b_i(\boldsymbol x_i)
    \end{equation*}
\end{remark}

\begin{definition}{\textbf{(Bathe Free Energy)}}
    We define it in the form of:
    \begin{equation*}
        \mathcal{F}_\text{bathe}(b) = \mathcal{E}_\text{bathe}(b) + \mathcal{H}_\text{bathe}(b) 
    \end{equation*}
    Both terms are approximated so that it corresponds to variational likelihood terms:
    \begin{itemize}
        \item Bathe average energy is the expected log-joint evaluate as though the pseudomarginal were correct:
        \begin{equation*}
            \mathcal{E}_\text{bathe}(b) = \sum_i\sum_{\boldsymbol x_i} b_i(\boldsymbol x_i)\log f_i(\boldsymbol x_i) + \sum_{ij}\sum_{\boldsymbol x_i\boldsymbol x_j} b_{ij}(\boldsymbol x_i, \boldsymbol x_j)\log f_{ij}(\boldsymbol x_i, \boldsymbol x_j)
        \end{equation*}
        \item Bathe entropy is the sum of pseudomarginal entropies corrected for pairwise (pseudo-)interaction, but neglecting higher-order dependence:
        \begin{equation*}
        \begin{aligned}
            \mathcal{H}_\text{bathe}(b) &= \sum_i H[b_i] - \sum_{ij}\operatorname{KL}[b_{ij} | b_ib_j] \\
            &= -\sum_i\sum_{\boldsymbol x_i} b_i(\boldsymbol x_i)\log b_i(\boldsymbol x_i) - \sum_{ij}\sum_{\boldsymbol x_i\boldsymbol x_j}b_{ij}(\boldsymbol x_i, \boldsymbol x_j)\log \frac{b_{ij}(\boldsymbol x_i, \boldsymbol x_j)}{b_i(\boldsymbol x_i)b_j(\boldsymbol x_j)}
        \end{aligned}
        \end{equation*}
    \end{itemize}
    On tree, both belief and the bathe entropy expression are correct i.e $\mathcal{F}_\text{bathe} = \mathcal{F}$. The update rule can be recoved from finding the fixed point. 
\end{definition}

\begin{proposition}{\textbf{(Fixed Point for Bathe Free Energy)}}
    The fixed point for Bathe free energy is:
    \begin{equation*}
    \begin{aligned}
        &b_i(\boldsymbol x_i) \propto f_i(\boldsymbol x_i)\prod_{j\in\operatorname{ne}(i)}\exp(-\xi_{ij}(\boldsymbol x_i)) \\
        &b_{ij}(\boldsymbol x_i, \boldsymbol x_j) \propto f_{ij}(\boldsymbol x_i, \boldsymbol x_j)b_i(\boldsymbol x_i)b_j(\boldsymbol x_j)\exp(\xi_{ij}(\boldsymbol x_i) + \xi_{ij}(\boldsymbol x_j)) \\
        &\exp(-\xi_{ij}(\boldsymbol x_i)) \propto \sum_{\boldsymbol x_j}f_{ij}(\boldsymbol x_i, \boldsymbol x_j)f_j(\boldsymbol x_j)\prod_{l\in\operatorname{ne}(j)\backslash i}\exp(-\xi_{ij}(\boldsymbol x_j)) \\
    \end{aligned}
    \end{equation*}
\end{proposition}
\begin{proof}
    We find the Lagragian with local consistency and normalization, which is given as:
    \begin{equation*}
    \begin{aligned}
        \mathcal{L} = \sum_{i}&\sum_{\boldsymbol x_i}b_i(\boldsymbol x_i)\log f_i(\boldsymbol x_i) + \sum_{ij}\sum_{\boldsymbol x_i\boldsymbol x_j}b_{ij}(\boldsymbol x_i, \boldsymbol x_j)\log f_{ij}(\boldsymbol x_i,\boldsymbol x_j) \\
        &- \sum_i\sum_{\boldsymbol x_i}b_i(\boldsymbol x_i)\log b_i(\boldsymbol x_i)-\sum_{ij}\sum_{\boldsymbol x_i\boldsymbol x_j}b_{ij}(\boldsymbol x_i, \boldsymbol x_j)\log\frac{b_{ij}(\boldsymbol x_i, \boldsymbol x_j)}{b_i(\boldsymbol x_i)b_j(\boldsymbol x_j)} \\
        &+ \sum_i\xi_i\bracka{\sum_{\boldsymbol x_i}b_i(\boldsymbol x_i)-1} \\
        &+ \sum_{ij}\brackb{ \sum_{\boldsymbol x_i}\xi_{ij}(\boldsymbol x_{i})\bracka{\sum_{\boldsymbol x_j}b_{ij}(\boldsymbol x_i, \boldsymbol x_j) - b_i(\boldsymbol x_i)} + \sum_{\boldsymbol x_j}\xi_{ij}(\boldsymbol x_j)\bracka{ \sum_{\boldsymbol x_i}b_{ij}(\boldsymbol x_i, \boldsymbol x_j) - b_j(\boldsymbol x_j) } }
    \end{aligned}
    \end{equation*}
    Setting the derivate to zero, which gives us the solution:
    \allowdisplaybreaks
    \begin{align*}
        &\begin{aligned}[t]
            \frac{\partial}{\partial b_i(\boldsymbol x_i)} &= \log f_i(\boldsymbol x_i) - \log b_i(\boldsymbol x_i) + \sum_{j\in\operatorname{ne}(j)}\sum_{\boldsymbol x_j}\frac{b_{ij}(\boldsymbol x_i, \boldsymbol x_j)}{b_i(\boldsymbol x_i)} + \xi_i - \sum_{j\in\operatorname{ne}(i)}\xi_{ij}(\boldsymbol x_i) + \const = 0 \\
            &\implies b_i(\boldsymbol x_i) \propto f_i(\boldsymbol x_i)\prod_{j\in\operatorname{ne}(i)} \exp(-\xi_{ij}(\boldsymbol x_i))
        \end{aligned} \\
        &\begin{aligned}[t]
            \frac{\partial f}{\partial b_{ij}(\boldsymbol x_i, \boldsymbol x_j)} &= \log f_{ij}(\boldsymbol x_i, \boldsymbol x_j) - \log b_{ij}(\boldsymbol x_i, \boldsymbol x_j) + \log b_i(\boldsymbol x_i)b_j(\boldsymbol x_j) + \xi_{ij}(\boldsymbol x_i) + \xi_{ji}(\boldsymbol x_j) + \const = 0 \\
            &\implies b_{ij}(\boldsymbol x_i, \boldsymbol x_j) \propto f_{ij}(\boldsymbol x_i, \boldsymbol x_j) b_i(\boldsymbol x_i)b_j(\boldsymbol x_j) \exp(\xi_{ij}(\boldsymbol x_i) + \xi_{ji}(\boldsymbol x_j))
        \end{aligned}
    \end{align*}
    To solve for $\xi_{ij}(\boldsymbol x_i)$ by enforcing the constant $\sum_{\boldsymbol x_j}b_{ij}(\boldsymbol x_i, \boldsymbol x_j) = b_i(\boldsymbol x_i)$ where we have:
    \begin{equation*}
    \begin{aligned}
        &\sum_{\boldsymbol x_j} b_{ij}(\boldsymbol x_i, \boldsymbol x_j) \propto \sum_{\boldsymbol x_j}f_{ij}(\boldsymbol x_i, \boldsymbol x_j)b_i(\boldsymbol x_i)b_j(\boldsymbol x_j)\exp(\xi_{ij}(\boldsymbol x_i) + \xi_{ji}(\boldsymbol x_j)) \\
        \implies& b_i(\boldsymbol x_i) \propto b_i(\boldsymbol x_i)\exp(\xi_{ij}(\boldsymbol x_i))\sum_{\boldsymbol x_j}f_{ij}(\boldsymbol x_i, \boldsymbol x_j)b_j(\boldsymbol x_j)\exp(\xi_{ji}(\boldsymbol x_j)) \\
        \implies& \begin{aligned}[t]
            \exp(-\xi_{ij}(\boldsymbol x_i)) &\propto \sum_{\boldsymbol x_j}f_{ij}(\boldsymbol x_i, \boldsymbol x_j)b_j(\boldsymbol x_j)\exp(\xi_{ji}(\boldsymbol x_j)) \\
            &= \sum_{\boldsymbol x_j}f_{ij}(\boldsymbol x_i, \boldsymbol x_j)f_j(\boldsymbol x_j)\prod_{l\in\operatorname{ne}(j)\backslash i}\exp(-\xi_{ji}(\boldsymbol x_j))
        \end{aligned}
    \end{aligned}
    \end{equation*}
\end{proof}

\begin{remark}{\textbf{(Interpretation of Results)}}
    Comparing with BP, we have the message to be of the form of $M_{j\rightarrow i}(\boldsymbol x_i) = \exp(-\xi_{ij}(\boldsymbol x_i))$. The fixed point for bathe free energy recovers the message passing rule:
    \begin{itemize}
        \item Stable Fixed point of loopy BP are stationary point of Bathe and local minimum of Bathe free energy. 
        \item For binary attractive netwrok: the Bathe free energy at fixed point of loopy BP provides an upperbound on the log partition function $\log P(\boldsymbol Z)$. 
        \item It is useful for learning undirected graphical model as it leads to lower bound on the log-likelihood. 
        \item Belief $b_i$ and $b_{ij}$ in loopy BP are only locally consistent pseudomarginal, not necessary consistent with marginal or the implied joint distribution. 
        \item Bathe free enerfy accounts for interaction between difference states, while variational free energy that assume independence. 
        \item The log series Plefka expansion of the log-partition $Z$: the variational energy form the first order while Bathe free energy contains higher term. 
        \item Loopy BP tends to significantly more accurate whenever it converges. 
    \end{itemize}
\end{remark}

\begin{remark}{\textbf{(Extensions and Variations)}}
    \begin{itemize}
        \item Generalized BP is a group variable together to threat their interaction exacely. 
        \item The algorithm can be derived so that the Bathe free energy at every step and thus guarantee the convergence. Similarly, convex alternative and we will converge to unique global maximum. 
        \item The treatment of loopy Viterbi or max-product algorithm is difference. 
    \end{itemize}
\end{remark}


% \begin{algorithm}[H]
%     \caption{$PSRO_{RN}$}
% 	\begin{algorithmic}[1]
% 	    \State \textbf{Input}: Initial Population $\mathcal{B}_1$
% 		\For {$i=1,2,\cdots, T$}
% 		    \State $p \leftarrow \text{Nash}(A_{\mathcal{B}_i})$
% 		    \For {agent $v_i$ with positive mass in $p_t$}
%                 \State $v_{i+1} \leftarrow \text{oracle}(v_i, \sum_{w \in \mathcal{B}_i} p[i](\phi_{v_i}(\cdot))_+)$
%             \EndFor
%             \State $\mathcal{B}_{i+1} = \mathcal{B} \cup \{v_{i+1} : \text{as updated above}\}$
% 		\EndFor
% 	\end{algorithmic} 
% \end{algorithm}

% \begin{table}[!h]
%   \centering
%   \begin{tabular}{lc}
%     \toprule
%     \textbf{Methods/Metrics}     & \textbf{Accuracy}  \\
%     \midrule
%     Logistic Regression          & $48.26 \pm 0.0f0$ \\
%     Support Vector Machine       & $48.91 \pm 0.00$  \\
%     Random Forest Classifier     & $44.38 \pm 1.57$  \\
%     \midrule
%     Multi-Dimensional ELO        & $34.51 \pm 3.12$  \\
%     TrueSkill\texttrademark      & $44.99 \pm 0.00$  \\
%     \bottomrule
%   \end{tabular}
  
%   \caption{}
  
%   \label{table}
% \end{table}

% \begin{AutoMultiColItemize}
%   \item Item 1
%   \item Item 2
%   \item Item 3
%   \item Item 4
%   \item Item 5
%   \item Item 6
% \end{AutoMultiColItemize}


% \bibliographystyle{plain}
% \bibliography{references}
\end{document}
